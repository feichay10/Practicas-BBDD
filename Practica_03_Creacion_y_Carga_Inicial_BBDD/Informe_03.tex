\documentclass[11pt]{report}

% Paquetes y configuraciones adicionales
\usepackage{graphicx}
\usepackage[export]{adjustbox}
\usepackage{caption}
\usepackage{float}
\usepackage{titlesec}
\usepackage{geometry}
\usepackage[hidelinks]{hyperref}
\usepackage{titling}
\usepackage{titlesec}
\usepackage{parskip}
\usepackage{wasysym}
\usepackage{tikzsymbols}
\usepackage{fancyvrb}
\usepackage{xurl}
\usepackage{hyperref}
\usepackage{subcaption}

\usepackage{listings}
\usepackage{xcolor}

\usepackage[spanish]{babel}

\newcommand{\subtitle}[1]{
  \posttitle{
    \par\end{center}
    \begin{center}\large#1\end{center}
    \vskip0.5em}
}

% Configura los márgenes
\geometry{
  left=2cm,   % Ajusta este valor al margen izquierdo deseado
  right=2cm,  % Ajusta este valor al margen derecho deseado
  top=3cm,
  bottom=3cm,
}

% Configuración de los títulos de las secciones
\titlespacing{\section}{0pt}{\parskip}{\parskip}
\titlespacing{\subsection}{0pt}{\parskip}{\parskip}
\titlespacing{\subsubsection}{0pt}{\parskip}{\parskip}

% Redefinir el formato de los capítulos y añadir un punto después del número
\makeatletter
\renewcommand{\@makechapterhead}[1]{%
  \vspace*{0\p@} % Ajusta este valor para el espaciado deseado antes del título del capítulo
  {\parindent \z@ \raggedright \normalfont
    \ifnum \c@secnumdepth >\m@ne
        \huge\bfseries \thechapter.\ % Añade un punto después del número
    \fi
    \interlinepenalty\@M
    #1\par\nobreak
    \vspace{10pt} % Ajusta este valor para el espacio deseado después del título del capítulo
  }}
\makeatother

% Configura para que cada \chapter no comience en una pagina nueva
\makeatletter
\renewcommand\chapter{\@startsection{chapter}{0}{\z@}%
    {-3.5ex \@plus -1ex \@minus -.2ex}%
    {2.3ex \@plus.2ex}%
    {\normalfont\Large\bfseries}}
\makeatother

% Configurar los colores para el código
\definecolor{codegreen}{rgb}{0,0.6,0}
\definecolor{codegray}{rgb}{0.5,0.5,0.5}
\definecolor{codepurple}{rgb}{0.58,0,0.82}
\definecolor{backcolour}{rgb}{0.95,0.95,0.92}

% Configurar el estilo para el código
\lstdefinestyle{mystyle}{
  backgroundcolor=\color{backcolour},   
  commentstyle=\color{codegreen},
  keywordstyle=\color{magenta},
  numberstyle=\tiny\color{codegray},
  stringstyle=\color{codepurple},
  basicstyle=\ttfamily\footnotesize,
  breakatwhitespace=false,         
  breaklines=true,                 
  captionpos=b,                    
  keepspaces=true,                 
  numbers=left,                    
  numbersep=5pt,                  
  showspaces=false,                
  showstringspaces=false,
  showtabs=false,                  
  tabsize=2
}

\begin{document}

% Portada del informe
\title{Practica 03. Creación y carga de la base de datos}
\subtitle{Bases de Datos}
\author{Cheuk Kelly Ng Pante (alu0101364544@ull.edu.es)}
\date{\today}

\maketitle

\pagestyle{empty} % Desactiva la numeración de página para el índice

% Índice
\tableofcontents

% Nueva página
\cleardoublepage

\pagestyle{plain} % Vuelve a activar la numeración de página
\setcounter{page}{1} % Reinicia el contador de página a 1

% Secciones del informe
% Capitulo 1
\chapter{Razonar el motivo de las definiciones hechas.}
En la creación de la tabla Departamento se pone como clave primaria el CD. A la
hora de crear la tabla Área se pone CAR como clave primaria y el campo CD se
indica que es una referencia a la tabla Departamento.
Durante la creación de la tabla Profesor se establece el campo DNI como clave
principal y CAR como clave foránea perteneciente la tabla Área.
En la creación de la tabla Asignatura se establece CAS como clave principal, se
indica que los campos A y T no puede ser nulo, también se indica que el campo
CUR coge valores entre 1 y 5, el campo CAR hace referencia a la tabla Área, y que
los campos CT, CP y CL tienen por defecto el valor 0.0.
Cuando se crea la tabla Plan\_Docente, se indican las claves DNI,CAS y FI como
claves principales, se establecen las clases CAS Y DNI como foráneas,
pertenecientes a las tablas Asignatura y Profesor respectivamente. Y se verifica
que la Fecha de Finalización(FF) es posterior a la Fecha de Inicio(FI)

\chapter{Contenido del archivo de spool PRACT3.lst}
\begin{verbatim}
  SQL> CREATE TABLE DEPARTAMENTO
  2  (CD NUMBER(2) PRIMARY KEY,
  3  D VARCHAR2(60));

Table created.

SQL> CREATE TABLE AREA
  2  (CAR NUMBER(3) PRIMARY KEY,
  3  AR VARCHAR2(60),
  4  CD NUMBER(2) REFERENCES DEPARTAMENTO ON DELETE CASCADE);

Table created.

SQL> CREATE TABLE PROFESOR
  2  (DNI NUMBER(8) PRIMARY KEY,
  3  P VARCHAR2(60),
  4  CAR NUMBER(3),
  5  CAT VARCHAR2(5),
  6  FOREIGN KEY (CAR) REFERENCES AREA ON DELETE SET NULL);

Table created.

SQL> CREATE TABLE ASIGNATURA
  2  (CAS NUMBER(3) PRIMARY KEY,
  3  A VARCHAR2(50) NOT NULL,
  4  T CHAR(4) NOT NULL,
  5  CUR NUMBER(1) CHECK (CUR BETWEEN 1 AND 5),
  6  CAR NUMBER(3) REFERENCES AREA ON DELETE SET NULL,
  7  CT NUMBER(3,1) DEFAULT 0.0,
  8  CP NUMBER(3,1) DEFAULT 0.0,
  9  CL NUMBER(3,1) DEFAULT 0.0);

Table created.

SQL> CREATE TABLE PLAN_DOCENTE
  2  (DNI NUMBER(8),
  3  CAS NUMBER(3),
  4  CTA NUMBER(3,1) DEFAULT 0.0,
  5  CPA NUMBER(3,1) DEFAULT 0.0,
  6  CLA NUMBER(3,1) DEFAULT 0.0,
  7  FI DATE DEFAULT SYSDATE,
  8  FF DATE,
  9  PRIMARY KEY (DNI, CAS, FI),
 10  FOREIGN KEY (CAS) REFERENCES ASIGNATURA ON DELETE CASCADE,
 11  FOREIGN KEY (DNI) REFERENCES PROFESOR ON DELETE CASCADE,
 12  CONSTRAINT PD_CK1 CHECK (CTA + CPA + CLA > 0.0),
 13  CONSTRAINT PD_CK2 CHECK (FF >= FI));

Table created.

SQL> spool off

SQL> INSERT INTO DEPARTAMENTO
  2  VALUES(1,'ANALISIS MATEMATICO');

1 row created.

SQL> INSERT INTO DEPARTAMENTO
  2  VALUES(2,'ASTROFISICA');

1 row created.

SQL> INSERT INTO DEPARTAMENTO
  2  VALUES(3,'ESTADISTICA, INVESTIGACION OPERATIVA Y COMPUTACION');

1 row created.

SQL> INSERT INTO DEPARTAMENTO
  2  VALUES(4,'MATEMATICA FUNDAMENTAL');

1 row created.

SQL> INSERT INTO AREA
  2  VALUES(1,'ALGEBRA',4);

1 row created.

SQL> INSERT INTO AREA
  2  VALUES(2,'ANALISIS MATEMATICO',1);

1 row created.

SQL> INSERT INTO AREA
  2  VALUES(3,'ASTRONOMIA Y ASTROFISICA',2);

1 row created.

SQL> INSERT INTO AREA
  2  VALUES(4,'CIENCIAS DE LA COMPUTACION E INTELIGENCIA ARTIFICIAL',3);

1 row created.

SQL> INSERT INTO AREA
  2  VALUES(4,'DIDACTICA DE LA MATEMATICA',1);
INSERT INTO AREA
*
ERROR at line 1:
ORA-00001: unique constraint (ALU0101364544.SYS_C00408097) violated 


SQL> INSERT INTO AREA
  2  VALUES(5,'DIDACTIVA DE LA MATEMATICA',1);

1 row created.

SQL> INSERT INTO AREA
  2  VALUES(6,'ESTADISTICA E INVESTIGACION OPERATIVA',3);

1 row created.

SQL> INSERT INTO AREA
  2  VALUES(7,'LENGUAJES Y SISTEMAS INFORMATICOS',3);

1 row created.

SQL> INSERT INTO AREA
  2  VALUES(8,'MATEMATICA APLICADA',1);

1 row created.

SQL> spool off

SQL> INSERT INTO PROFESOR
  2  VALUES(1111,'JUAN',6'CU');
VALUES(1111,'JUAN',6'CU')
                    *
ERROR at line 2:
ORA-00917: missing comma 


SQL> INSERT INTO PROFESOR
  2  VALUES(1111,'JUAN',6,'CU');

1 row created.

SQL> INSERT INTO PROFESOR
  2  VALUES(2222,'CARLOS',7,'TU');

1 row created.

SQL> INSERT INTO PROFESOR
  2  VALUES(3333,'PEDRO',4,'TEU');

1 row created.

SQL> INSERT INTO PROFESOR
  2  VALUES(4444,'MARIA',7,'TU');

1 row created.

SQL> INSERT INTO PROFESOR
  2  VALUES(5555,'IVAN',1,'CEU');

1 row created.

SQL> INSERT INTO PROFESOR
  2  VALUES(6666,'CARMEN',3,'CD');

1 row created.

SQL> INSERT INTO PROFESOR
  2  VALUES(7777,'MARIO',2,'TU');

1 row created.

SQL> VALUES(8888,'FRANCISCO',5,'TU');
SP2-0734: unknown command beginning "VALUES(888..." - rest of line ignored.
SQL> INSERT INTO PROFESOR
  2  VALUES(8888,'FRANCISCO',5,'TU');

1 row created.

SQL> INSERT INTO PROFESOR
  2  VALUES(9999,'ANGELA',8,'TEU);
ERROR:
ORA-01756: quoted string not properly terminated 


SQL> INSERT INTO PROFESOR
  2  VALUES(9999,'ANGELA',8,'TEU');

1 row created.

SQL> INSERT INTO PROFESOR
  2  VALUES(1010,'DAVID',4,'TU');

1 row created.

SQL> INSERT INTO PROFESOR
  2  VALUES(2020,SOLEDAD,7,'CU');
VALUES(2020,SOLEDAD,7,'CU')
            *
ERROR at line 2:
ORA-00984: column not allowed here 


SQL> INSERT INTO PROFESOR
  2  VALUES(2020,'SOLEDAD',7,'CU');

1 row created.

SQL> INSERT INTO PROFESOR
  2  VALUES(3030,'JOSE MANUEL',6,'TEU');

1 row created.

SQL> spool off
SQL> INSERT INTO ASIGNATURA
  2  VALUES(1,'BASE DE DATOS','GII',3,7,3,1.5,1.5);

1 row created.

SQL> COMMIT
  2  ;

Commit complete.

SQL> INSERT INTO ASIGNATURA
  2  VALUES(2,'INTELIGENCIA ARTIFICIAL','GII',3,4,1.5,1.5,3);

1 row created.

SQL> spool off

SQL> INSERT INTO ASIGNATURA
  2  VALUES(3,'ALMACENES DE DATOS','MII',1,7,1.5,0,1.5);

1 row created.

SQL> INSERT INTO ASIGNATURA
  2  VALUES(4,'MINERIA DE DATOS','MII',1,7,1.5,0,1.5);

1 row created.

SQL> INSERT INTO ASIGNATURA
  2  VALUES(5,'INFORMATICA BASICA','GII',1,7,3,1.5,1.5);

1 row created.

SQL> INSERT INTO ASIGNATURA
  2  VALUES(6,'ALGEBRA','GII',1,1,3,3,0);

1 row created.

SQL> INSERT INTO ASIGNATURA
  2  VALUES(7,'CALCULO','GII',1,8,3,3,0);

1 row created.

SQL> INSERT INTO ASIGNATURA
  2  VALUES(8,'OPTIMIZACION','GII',1,6,3,1.5,1.5);

1 row created.

SQL> INSERT INTO ASIGNATUA
  2  VALUES(9,'GESTION DE RIESGOS','GII',3,4,3,0,3);
INSERT INTO ASIGNATUA
            *
ERROR at line 1:
ORA-00942: table or view does not exist


SQL> INSERT INTO ASIGNATURA
  2  VALUES(9,'GESTION DE RIESGOS','GII',3,4,3,0,3);

1 row created.

SQL> INSERT INTO ASIGNATURA
  2  VALUES(10,'ASTRONOMIA','GF',2,3,3,1.5,1.5);

1 row created.

SQL> INSERT INTO ASIGNATURA
  2  VALUES(11,'DIDACTICA DE LA MATEMATICA','GM',2,5,6,0,0);

1 row created.

SQL> INSERT INTO ASIGNATURA
  2  VALUES(12,'ANALISIS COMPLEJO','GM',4,2,4.5,3,0);

1 row created.

SQL> spool off

SQL> INSERT INTO PLAN_DOCENTE
  2  VALUES(4444, 1, 3, 1.5, 1.5, '01-SEP-11', NULL);

1 row created.

SQL> INSERT INTO PLAN_DOCENTE
  2  VALUES(4444,4,1.5,0,1.5'01-SEP-08','31-AUG-10');
VALUES(4444,4,1.5,0,1.5'01-SEP-08','31-AUG-10')
                       *
ERROR at line 2:
ORA-00917: missing comma 


SQL> INSERT INTO PLAN_DOCENTE
  2  VALUES(4444,4,1.5,0,1.5,'01-SEP-08','31-AUG-10');

1 row created.

SQL> INSERT INTO PLAN_DOCENTE
  2  VALUES(4444,5,3,0,0,'01-SEP-10',NULL);

1 row created.

SQL> INSERT INTO PLAN_DOCENTE
  2  VALUES(1111,8,3,1.5,1.5,'01-SEP-07','31-AUG-09');

1 row created.

SQL> INSERT INTO PLAN_DOCENTE
  2  VALUES(1111,8,3,0,0,'01-SEP-09',NULL);

1 row created.

SQL> INSERT INTO PLAN_DOCENTE
  2  VALUES(3030,8,0,1.5,1.5,'01-SEP-09',NULL);

1 row created.

SQL> INSERT INTO PLAN_DOCENTE
  2  VALUES(2222,4,1.5,0,1.5,'01-SEP-09',NULL);

1 row created.

SQL> INSERT INTO PLAN_DOCENTE
  2  VALUES(2222,3,1.5,0,1.5,'01-SEP-06','31-AUG-07');

1 row created.

SQL> INSERT INTO PLAN_DOCENTE
  2  VALUES(1010,2,1.5,1.5,3,'01-SEP-05','31-AUG-08');

1 row created.

SQL> INSERT INTO PLAN_DOCENTE
  2  VALUES(3333,2,1.5,1.5,3,'01-SEP-08',NULL);

1 row created.

SQL> INSERT INTO PLAN_DOCENTE
  2  VALUES(1010,9,3,0,3,'01-SEP-08','31-AUG-09');

1 row created.

SQL> INSERT INTO PLAN_DOCENTE
  2  VALUES(1010,9,1.5,0,1.5,'01-SEP-09',NULL)
  3  ;

1 row created.

SQL> INSERT INTO PLAN_DOCENTE
  2  VALUES(9999,7,3,3,0,'01-SEP-10',NULL);

1 row created.

SQL> INSERT INTO PLAN_DOCENTE
  2  VALUES(5555,6,3,3,0,'31-MAR-10',NULL);

1 row created.

SQL> INSERT INTO PLAN_DOCENTE
  2  VALUES(6666,10,3,1.5,1.5,'01-SEP-08','31-AUG-11');

1 row created.

SQL> INSERT INTO PLAN_DOCENTE
  2  VALUES(8888,11,6,0,0,'01-SEP-09',NULL)
  3  ;

1 row created.

SQL> INSERT INTO PLAN_DOCENTE
  2  VALUES(2020,3,1.5,0,1.5,'01-SEP-08',NULL);

1 row created.

SQL> INSERT INTO PLAN_DOCENTE
  2  VALUES(7777,12,4.5,3,0,'01-SEP-10',NULL);

1 row created.

SQL> INSERT INTO PLAN_DOCENTE
  2  VALUES(3333,9,1.5,0,1.5,'01-SEP-09',NULL);

1 row created.

SQL> spool off
SQL> select cd, d from departamento;

        CD D
---------- ------------------------------------------------------------
         1 ANALISIS MATEMATICO
         2 ASTROFISICA
         3 ESTADISTICA, INVESTIGACION OPERATIVA Y COMPUTACION
         4 MATEMATICA FUNDAMENTAL

SQL> select * from area;

       CAR AR
---------- ------------------------------------------------------------
        CD
----------
         1 ALGEBRA
         4

         2 ANALISIS MATEMATICO
         1

         3 ASTRONOMIA Y ASTROFISICA
         2


       CAR AR
---------- ------------------------------------------------------------
        CD
----------
         4 CIENCIAS DE LA COMPUTACION E INTELIGENCIA ARTIFICIAL
         3

         5 DIDACTICA DE LA MATEMATICA
         1

         6 ESTADISTICA E INVESTIGACION OPERATIVA
         3


       CAR AR
---------- ------------------------------------------------------------
        CD
----------
         7 LENGUAJES Y SISTEMAS INFORMATICOS
         3

         8 MATEMATICA APLICADA
         1


8 rows selected.

SQL> select * from profesor;

       DNI P
---------- ------------------------------------------------------------
       CAR CAT
---------- -----
      1111 JUAN
         6 CU

      2222 CARLOS
         7 TU

      3333 PEDRO
         4 TEU


       DNI P
---------- ------------------------------------------------------------
       CAR CAT
---------- -----
      4444 MARIA
         7 TU

      5555 IVAN
         1 CEU

      6666 CARMEN
         3 CD


       DNI P
---------- ------------------------------------------------------------
       CAR CAT
---------- -----
      7777 MARIO
         2 TU

      8888 FRANCISCO
         5 TU

      9999 ANGELA
         8 TEU


       DNI P
---------- ------------------------------------------------------------
       CAR CAT
---------- -----
      1010 DAVID
         4 TU

      2020 SOLEDAD
         7 CU

      3030 JOSE MANUEL
         6 TEU


12 rows selected.

SQL> select * from asignatura;

       CAS A                                                  T           CUR
---------- -------------------------------------------------- ---- ----------
       CAR         CT         CP         CL
---------- ---------- ---------- ----------
         1 BASES DE DATOS                                     GII           3
         7          3        1.5        1.5

         2 INTELIGENCIA ARTIFICIAL                            GII           3
         4        1.5        1.5          3

         3 ALMACENES DE DATOS                                 MII           1
         7        1.5          0        1.5


       CAS A                                                  T           CUR
---------- -------------------------------------------------- ---- ----------
       CAR         CT         CP         CL
---------- ---------- ---------- ----------
         4 MINERIA DE DATOS                                   MII           1
         7        1.5          0        1.5

         5 INFORMATICA BASICA                                 GII           1
         7          3        1.5        1.5

         6 ALGEBRA                                            GII           1
         1          3          3          0 


       CAS A                                                  T           CUR
---------- -------------------------------------------------- ---- ----------
       CAR         CT         CP         CL
---------- ---------- ---------- ----------
         7 CALCULO                                            GII           1
         8          3          3          0 

         8 OPTIMIZACION                                       GII           1
         6          3        1.5        1.5

        9 GESTION DE RIESGOS                                 GII           3
         4          3          0          3


       CAS A                                                  T           CUR
---------- -------------------------------------------------- ---- ----------
       CAR         CT         CP         CL
---------- ---------- ---------- ----------
        10 ASTRONOMIA                                         GF            2
         3          3        1.5        1.5

        11 DIDACTICA DE LA MATEMATICA                         GM            2
         5          6          0          0 

        12 ANALISIS COMPLEJO                                  GM            4
         2        4.5          3          0 


12 rows selected.

SQL> select * from plan_docente;

       DNI        CAS        CTA        CPA        CLA FI        FF
---------- ---------- ---------- ---------- ---------- --------- ---------
      4444          1          3        1.5        1.5 01-SEP-11
      4444          4        1.5          0        1.5 01-SEP-08 31-AUG-10
      4444          5          3          0          0 01-SEP-10
      1111          8          3        1.5        1.5 01-SEP-07 31-AUG-09
      1111          8          3          0          0 01-SEP-09
      3030          8          0        1.5        1.5 01-SEP-09
      2222          4        1.5          0        1.5 01-SEP-09
      2222          3        1.5          0        1.5 01-SEP-06 31-AUG-07 
      1010          2        1.5        1.5          3 01-SEP-05 31-AUG-08
      3333          2        1.5        1.5          3 01-SEP-08
      1010          9          3          0          3 01-SEP-08 31-AUG-09

       DNI        CAS        CTA        CPA        CLA FI        FF
---------- ---------- ---------- ---------- ---------- --------- ---------
      1010          9        1.5          0        1.5 01-SEP-09
      9999          7          3          3          0 01-SEP-10
      5555          6          3          3          0 31-MAR-10
      6666         10          3        1.5        1.5 01-SEP-08 31-AUG-11
      8888         11          6          0          0 01-SEP-09
      2020          3        1.5          0        1.5 01-SEP-08
      7777         12        4.5          3          0 01-SEP-10
      3333          9        1.5          0        1.5 01-SEP-09

19 rows selected.

SQL> spool off
\end{verbatim}

\end{document}