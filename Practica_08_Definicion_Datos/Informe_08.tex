\documentclass[11pt]{report}

% Paquetes y configuraciones adicionales
\usepackage{graphicx}
\usepackage[export]{adjustbox}
\usepackage{caption}
\usepackage{float}
\usepackage{titlesec}
\usepackage{geometry}
\usepackage[hidelinks]{hyperref}
\usepackage{titling}
\usepackage{titlesec}
\usepackage{parskip}
\usepackage{wasysym}
\usepackage{tikzsymbols}
\usepackage{fancyvrb}
\usepackage{xurl}
\usepackage{hyperref}
\usepackage{subcaption}

\usepackage{listings}
\usepackage{xcolor}

\usepackage[spanish]{babel}

\newcommand{\subtitle}[1]{
  \posttitle{
    \par\end{center}
    \begin{center}\large#1\end{center}
    \vskip0.5em}
}

% Configura los márgenes
\geometry{
  left=2cm,   % Ajusta este valor al margen izquierdo deseado
  right=2cm,  % Ajusta este valor al margen derecho deseado
  top=3cm,
  bottom=3cm,
}

% Configuración de los títulos de las secciones
\titlespacing{\section}{0pt}{\parskip}{\parskip}
\titlespacing{\subsection}{0pt}{\parskip}{\parskip}
\titlespacing{\subsubsection}{0pt}{\parskip}{\parskip}

% Redefinir el formato de los capítulos y añadir un punto después del número
\makeatletter
\renewcommand{\@makechapterhead}[1]{%
  \vspace*{0\p@} % Ajusta este valor para el espaciado deseado antes del título del capítulo
  {\parindent \z@ \raggedright \normalfont
    \ifnum \c@secnumdepth >\m@ne
        \huge\bfseries \thechapter.\ % Añade un punto después del número
    \fi
    \interlinepenalty\@M
    #1\par\nobreak
    \vspace{10pt} % Ajusta este valor para el espacio deseado después del título del capítulo
  }}
\makeatother

% Configura para que cada \chapter no comience en una pagina nueva
\makeatletter
\renewcommand\chapter{\@startsection{chapter}{0}{\z@}%
    {-3.5ex \@plus -1ex \@minus -.2ex}%
    {2.3ex \@plus.2ex}%
    {\normalfont\Large\bfseries}}
\makeatother

% Configurar los colores para el código
\definecolor{codegreen}{rgb}{0,0.6,0}
\definecolor{codegray}{rgb}{0.5,0.5,0.5}
\definecolor{codepurple}{rgb}{0.58,0,0.82}
\definecolor{backcolour}{rgb}{0.95,0.95,0.92}

% Configurar el estilo para el código
\lstdefinestyle{mystyle}{
  backgroundcolor=\color{backcolour},   
  commentstyle=\color{codegreen},
  keywordstyle=\color{magenta},
  numberstyle=\tiny\color{codegray},
  stringstyle=\color{codepurple},
  basicstyle=\ttfamily\footnotesize,
  breakatwhitespace=false,         
  breaklines=true,                 
  captionpos=b,                    
  keepspaces=true,                 
  numbers=left,                    
  numbersep=5pt,                  
  showspaces=false,                
  showstringspaces=false,
  showtabs=false,                  
  tabsize=2
}

\begin{document}

% Portada del informe
\title{Practica 08. Definición de datos}
\subtitle{Bases de Datos}
\author{Cheuk Kelly Ng Pante (alu0101364544@ull.edu.es)}
\date{\today}

\maketitle

\pagestyle{empty} % Desactiva la numeración de página para el índice

% Índice
\tableofcontents

% Nueva página
\cleardoublepage

\pagestyle{plain} % Vuelve a activar la numeración de página
\setcounter{page}{1} % Reinicia el contador de página a 1

% Secciones del informe
%Pregunta 1:
\chapter{Crear una tabla llamada TITULACION con los atributos T (nombre de titulación) y FAC (nombre de facultad). La clave primaria es T. Especifica los tipos de datos que creas convenientes y razona tu elección.}
\begin{itemize}
  \item Consulta:
  \begin{verbatim}
SQL> CREATE TABLE TITULACION
  2  (T CHAR(4) NOT NULL,
  3  FAC VARCHAR(60),
  4  PRIMARY KEY(T));
  \end{verbatim}
  \item{Resultado:}
  \begin{verbatim}
Table created.
  \end{verbatim}
  \item{Explicación:} Los tipos de datos se han elegido en el caso de T, porque ya este parámetro existía en la tabla asignatura, y en cuanto a FAC, he elegido varchar porque es un campo que puede tener una longitud variable.
\end{itemize}

%Pregunta 2:
\chapter{Insertar, mediante una única instrucción, en la tabla TITULACION, los nombres de las TITULACION obtenidos a partir de la tabla ASIGNATURA.}
\begin{itemize}
  \item Consulta:
  \begin{verbatim}
SQL> INSERT INTO TITULACION(T)
  2  SELECT DISTINCT T
  3  FROM ASIGNATURA;
  \end{verbatim}
  \item{Resultado:}
  \begin{verbatim}
4 rows created.
  \end{verbatim}
\end{itemize}

%Pregunta 3:
\chapter{Listar todos los registros de la tabla TITULACION.}
\begin{itemize}
  \item Consulta:
  \begin{verbatim}
SQL> SELECT * FROM TITULACION;
  \end{verbatim}

  \newpage

  \item{Resultado:}
  \begin{verbatim}
T    FAC                                                                        
---- ------------------------------------------------------------               
GF                                                                              
GM                                                                              
GII                                                                             
MII 
  \end{verbatim}
\end{itemize}

%Pregunta 4:
\chapter{Rellenar convenientemente el campo FAC de la tabla TITULACION.}
\begin{itemize}
  \item Consulta:
  \begin{verbatim}
SQL> UPDATE TITULACION
  2  SET FAC = 'MATEMATICAS Y FISICA'
  3  WHERE T = 'GF';
  
SQL> UPDATE TITULACION
  2  SET FAC = 'MATEMATICAS Y FISICA'
  3  WHERE T = 'GM';
  
SQL> UPDATE TITULACION
  2  SET FAC = 'ESIT'
  3  WHERE T = 'GII';
  
SQL> UPDATE TITULACION
  2  SET FAC = 'ESIT'
  3  WHERE T = 'MII';
  \end{verbatim}
  \item{Resultado:}
  \begin{verbatim}
1 row updated.

1 row updated.

1 row updated.

1 row updated.
  \end{verbatim}
\end{itemize}

\newpage

%Pregunta 5:
\chapter{Crear un sinónimo denominado TIT para la tabla TITULACION.}
\begin{itemize}
  \item Consulta:
  \begin{verbatim}
SQL> CREATE SYNONYM TIT
  2  FOR TITULACION;
  \end{verbatim}
  \item{Resultado:}
  \begin{verbatim}
Synonym created.
  \end{verbatim}
\end{itemize}

%Pregunta 6:
\chapter{Listar todos los registros de TIT.}
\begin{itemize}
  \item Consulta:
  \begin{verbatim}
SQL> SELECT * FROM TIT;
  \end{verbatim}
  \item{Resultado:}
  \begin{verbatim}
T    FAC                                                                        
---- ------------------------------------------------------------               
GF   MATEMATICAS Y FISICA                                                       
GM   MATEMATICAS Y FISICA                                                       
GII  ESIT                                                                       
MII  ESIT   
  \end{verbatim}
\end{itemize}

%Pregunta 7:
\chapter{Añadir una condición de integridad referencial entre el atributo T de la tabla ASIGNATURAS y el atributo T de la tabla TITULACION con borrado en cascada.}
\begin{itemize}
  \item Consulta:
  \begin{verbatim}
SQL> ALTER TABLE ASIGNATURA
  2  ADD FOREIGN KEY(T) REFERENCES TITULACION (T) ON DELETE CASCADE;
  \end{verbatim}
  \item{Resultado:}
  \begin{verbatim}
Table altered.
  \end{verbatim}
\end{itemize}

\newpage

%Pregunta 8:
\chapter{Incrementar en dos caracteres la longitud del campo T en la tabla TITULACION.}
\begin{itemize}
  \item Consulta:
  \begin{verbatim}
SQL> ALTER TABLE TITULACION
  2  MODIFY T CHAR(6);
  \end{verbatim}
  \item{Resultado:}
  \begin{verbatim}
ALTER TABLE TITULACION
*
ERROR at line 1:
ORA-02292: integrity constraint (ALU0101364544.SYS_C00487529) violated - child 
record found 
  \end{verbatim}
\end{itemize}

%Pregunta 9:
\chapter{Crear una vista, llamada ‘VISTA1’, sobre la tabla ASIGNATURAS, con los campos (CAS, A, T, CUR, CAR).}
\begin{itemize}
  \item Consulta:
  \begin{verbatim}
SQL> CREATE VIEW VISTA1
  2  AS SELECT CAS, A, T, CUR, CAR
  3  FROM ASIGNATURA;
  \end{verbatim}
  \item{Resultado:}
  \begin{verbatim}
View created.
  \end{verbatim}
\end{itemize}

\newpage

%Pregunta 10:
\chapter{Listar todas las t-uplas de la vista ‘VISTA1’.}
\begin{itemize}
  \item Consulta:
  \begin{verbatim}
SQL> SELECT * FROM VISTA1;
  \end{verbatim}
  \item{Resultado:}
\end{itemize}
  \begin{verbatim}
       CAS A                                                  T           CUR        CAR                                                                                                                
---------- -------------------------------------------------- ---- ---------- ----------                                                                                                                
         1 BASES DE DATOS                                     GII           3          7                                                                                                                
         2 INTELIGENCIA ARTIFICIAL                            GII           3          4                                                                                                                
         3 ALMACENES DE DATOS                                 MII           1          7                                                                                                                
         4 MINERIA DE DATOS                                   MII           1          7                                                                                                                
         5 INFORMATICA BASICA                                 GII           1          7                                                                                                                
         6 ALGEBRA                                            GII           1          1                                                                                                                
         7 CALCULO                                            GII           1          8                                                                                                                
         8 OPTIMIZACION                                       GII           1          6                                                                                                                
         9 GESTION DE RIESGOS                                 GII           3          4                                                                                                                
        10 ASTRONOMIA                                         GF            2          3                                                                                                                
        11 DIDACTICA DE LA MATEMATICA                         GM            2          5                                                                                                                
        12 ANALISIS COMPLEJO                                  GM            4          2                                                                                                                

12 rows selected.
  \end{verbatim}

%Pregunta 11:
\chapter{Insertar la t-upla (13, ‘ESTADISTICA’, GII, 3, 7) en VISTA1.}
\begin{itemize}
  \item Consulta:
  \begin{verbatim}
SQL> INSERT INTO VISTA1
  2  VALUES(13, 'ESTADISTICA', 'GII', 3, 7);
  \end{verbatim}
  \item{Resultado:}
  \begin{verbatim}
1 row created.
  \end{verbatim}
\end{itemize}

%Pregunta 12:
\chapter{Listar todas las t-uplas de la vista ‘VISTA1’.}
\begin{itemize}
  \item Consulta:
  \begin{verbatim}
SQL> SELECT * FROM VISTA1;
  \end{verbatim}

  \newpage

  \item{Resultado:}
\end{itemize}
  \begin{verbatim}
       CAS A                                                  T           CUR        CAR                                                                                                                
---------- -------------------------------------------------- ---- ---------- ----------                                                                                                                
         1 BASES DE DATOS                                     GII           3          7                                                                                                                
         2 INTELIGENCIA ARTIFICIAL                            GII           3          4                                                                                                                
         3 ALMACENES DE DATOS                                 MII           1          7                                                                                                                
         4 MINERIA DE DATOS                                   MII           1          7                                                                                                                
         5 INFORMATICA BASICA                                 GII           1          7                                                                                                                
         6 ALGEBRA                                            GII           1          1                                                                                                                
         7 CALCULO                                            GII           1          8                                                                                                                
         8 OPTIMIZACION                                       GII           1          6                                                                                                                
         9 GESTION DE RIESGOS                                 GII           3          4                                                                                                                
        10 ASTRONOMIA                                         GF            2          3                                                                                                                
        11 DIDACTICA DE LA MATEMATICA                         GM            2          5                                                                                                                
        12 ANALISIS COMPLEJO                                  GM            4          2                                                                                                                
        13 ESTADISTICA                                        GII           3          7                                                                                                                

13 rows selected.
  \end{verbatim}

%Pregunta 13:
\chapter{Listar todas las t-uplas de la tabla ASIGNATURAS.}
\begin{itemize}
  \item Consulta:
  \begin{verbatim}
SQL> SELECT * FROM ASIGNATURA;
  \end{verbatim}
  \item{Resultado:}
\end{itemize}
  \begin{verbatim}
 CAS A                                    T    CUR     CAR         CT         CP        CL                                                           
---- --------------------------------- ---- ------ ------- ---------- ---------- ---------                                                          
         1 BASES DE DATOS               GII      3       7          3       1.5        1.5                                                           
         2 INTELIGENCIA ARTIFICIAL      GII      3       4        1.5       1.5          3                                                           
         3 ALMACENES DE DATOS           MII      1       7        1.5         0        1.5                                                           
         4 MINERIA DE DATOS             MII      1       7        1.5         0        1.5                                                           
         5 INFORMATICA BASICA           GII      1       7          3        1.5       1.5                                                           
         6 ALGEBRA                      GII      1       1          3          3         0                                                           
         7 CALCULO                      GII      1       8          3          3         0                                                           
         8 OPTIMIZACION                 GII      1       6          3        1.5       1.5                                                           
         9 GESTION DE RIESGOS           GII      3       4          3          0         3                                                           
        10 ASTRONOMIA                    GF      2       3          3        1.5       1.5                                                           
        11 DIDACTICA DE LA MATEMATICA    GM      2       5          6          0         0                                                                                                                      
        12 ANALISIS COMPLEJO             GM      4       2        4.5          3         0                                                           
        13 ESTADISTICA                  GII      3       7          0          0         0                                                           

13 rows selected.
  \end{verbatim}


%Pregunta 14:
\chapter{Modificar el campo CAR en ‘VISTA1’ de la t-upla con CAS 13. El nuevo valor es 6.}
\begin{itemize}
  \item Consulta:
  \begin{verbatim}
SQL> UPDATE VISTA1
  2  SET CAR = 6
  3  WHERE CAS = 13;
  \end{verbatim}
  \item{Resultado:}
  \begin{verbatim}
1 row updated.
  \end{verbatim}
\end{itemize}

%Pregunta 15:
\chapter{Listar todas las t-uplas de la vista ‘VISTA1’.}
\begin{itemize}
  \item Consulta:
  \begin{verbatim}
SQL> SELECT * FROM VISTA1;
  \end{verbatim}
  \item{Resultado:}
\end{itemize}
  \begin{verbatim}
       CAS A                                                  T           CUR        CAR                                                                                                                
---------- -------------------------------------------------- ---- ---------- ----------                                                                                                                
         1 BASES DE DATOS                                     GII           3          7                                                                                                                
         2 INTELIGENCIA ARTIFICIAL                            GII           3          4                                                                                                                
         3 ALMACENES DE DATOS                                 MII           1          7                                                                                                                
         4 MINERIA DE DATOS                                   MII           1          7                                                                                                                
         5 INFORMATICA BASICA                                 GII           1          7                                                                                                                
         6 ALGEBRA                                            GII           1          1                                                                                                                
         7 CALCULO                                            GII           1          8                                                                                                                
         8 OPTIMIZACION                                       GII           1          6                                                                                                                
         9 GESTION DE RIESGOS                                 GII           3          4                                                                                                                
        10 ASTRONOMIA                                         GF            2          3                                                                                                                
        11 DIDACTICA DE LA MATEMATICA                         GM            2          5                                                                                                                
        12 ANALISIS COMPLEJO                                  GM            4          2                                                                                                                
        13 ESTADISTICA                                        GII           3          6                                                                                                                

13 rows selected.
  \end{verbatim}

%Pregunta 16:
\chapter{¿Qué conclusiones sacas sobre las actualizaciones de ‘VISTA1’?}
Se pueden hacer actualizaciones en la vista, pero no se pueden hacer en la tabla original.

%Pregunta 17:
\chapter{Crear una vista, llamada ‘VISTA2’, sobre la tabla ’PLAN\_DOCENTE’ y la vista ‘VISTA1’ que contenga los datos (DNI, A, T, CUR, CAR) relativos a las asignaturas impartidas por un profesor especificado por su DNI.}
\begin{itemize}
  \item Consulta:
  \begin{verbatim}
SQL> CREATE VIEW VISTA2
  2  AS SELECT DNI, A, T, CUR, CAR
  3  FROM VISTA1 NATURAL JOIN PLAN_DOCENTE
  4  WHERE DNI = 1111;
  \end{verbatim}
  \item{Resultado:}
  \begin{verbatim}
View created.
  \end{verbatim}
\end{itemize}

%Pregunta 18:
\chapter{Listar todas las t-uplas de la vista ‘VISTA2’.}
\begin{itemize}
  \item Consulta:
  \begin{verbatim}
SQL> SELECT * FROM VISTA2;
  \end{verbatim}
  \item{Resultado:}
  \begin{verbatim}
       DNI A                                                  T           CUR        CAR                                                                                                                                                                                                                    
---------- -------------------------------------------------- ---- ---------- ----------                                                                                                                                                                                                                    
      1111 OPTIMIZACION                                       GII           1          6                                                                                                                                                                                                                    
      1111 OPTIMIZACION                                       GII           1          6                                                                                                                                                                                                                        
  \end{verbatim}
\end{itemize}

%Pregunta 19:
\chapter{Modifica el campo A de alguna t-upla específica de ‘VISTA2’. Interpreta el resultado.}
\begin{itemize}
  \item Consulta:
  \begin{verbatim}
SQL> UPDATE VISTA2
  2  SET A = 'OPTIMIZACION Y GRAFOS'
  3  WHERE A = 'OPTIMIZACION';
  \end{verbatim}

  \newpage

  \item{Resultado:}
  \begin{verbatim}
SET A = 'OPTIMIZACION Y GRAFOS'
    *
ERROR at line 2:
ORA-01779: cannot modify a column which maps to a non key-preserved table 
  \end{verbatim}
  \item{Explicación:} No deja ya que se está modificando un valor de varias tablas a la vez.
\end{itemize}

%Pregunta 20:
\chapter{Obtener el esquema de la vista ‘VISTA2’.}
\begin{itemize}
  \item Consulta:
  \begin{verbatim}
SQL> DESCRIBE VISTA2;
  \end{verbatim}
  \item{Resultado:}
  \begin{verbatim}
Name                                      Null?    Type
----------------------------------------- -------- ----------------------------
DNI                                       NOT NULL NUMBER(8)
A                                         NOT NULL VARCHAR2(50)
T                                         NOT NULL CHAR(4)
CUR                                                NUMBER(1)
CAR                                                NUMBER(3)
  \end{verbatim}
\end{itemize}

%Pregunta 21:
\chapter{Eliminar la vista ‘VISTA1’.}
\begin{itemize}
  \item Consulta:
  \begin{verbatim}
SQL> DROP VIEW VISTA1;
  \end{verbatim}
  \item{Resultado:}
  \begin{verbatim}
View dropped.
  \end{verbatim}
\end{itemize}

\newpage

%Pregunta 22:
\chapter{Listar todas las t-uplas de la vista ‘VISTA2’.}
\begin{itemize}
  \item Consulta:
  \begin{verbatim}
SQL> SELECT * FROM VISTA2;
  \end{verbatim}
  \item{Resultado:}
  \begin{verbatim}
SELECT * FROM VISTA2
              *
ERROR at line 1:
ORA-04063: view "ALU0101364544.VISTA2" has errors 
  \end{verbatim}
\end{itemize}

%Pregunta 23:
\chapter{Deshacer los cambios. ¿Se ha recuperado la vista VISTA1? ¿A qué crees que se debe?}
\begin{itemize}
  \item Consulta:
  \begin{verbatim}
SQL> ROLLBACK;
  \end{verbatim}
  \item{Resultado:}
  \begin{verbatim}
Rollback complete.
  \end{verbatim}
  \item{Explicación:} No se ha recuperado porque antes se ha usado el DROP y no se puede recuperar
\end{itemize}

%Pregunta 24:
\chapter{Crear un índice, llamado ‘INDICE1’, sobre el atributo P de la tabla ‘PROFESORES’. Listar los DNI de los profesores con nombre ‘JUAN’.}
\begin{itemize}
  \item Consulta:
  \begin{verbatim}
SQL> CREATE INDEX INDICE1
  2  ON PROFESOR(P);

SQL> SELECT DNI
  2  FROM PROFESOR
  3  WHERE P='JUAN';
  \end{verbatim}
  \item{Resultado:}
  \begin{verbatim}
       DNI                                                                      
----------                                                                      
      1111  
  \end{verbatim}
\end{itemize}

%Pregunta 25:
\chapter{Eliminar el índice ‘INDICE1’.}
\begin{itemize}
  \item Consulta:
  \begin{verbatim}
SQL> DROP INDEX INDICE1;
  \end{verbatim}
  \item{Resultado:}
  \begin{verbatim}
Index dropped.
  \end{verbatim}
\end{itemize}

%Pregunta 26:
\chapter{Crear un índice, llamado ‘INDICE2’, sobre el atributo CAR de la vista ‘VISTA1’. ¿Qué interpretas?}
\begin{itemize}
  \item Consulta:
  \begin{verbatim}
SQL> CREATE INDEX INDICE2
  2  ON VISTA1(CAR);
  \end{verbatim}
  \item{Resultado:}
  \begin{verbatim}
ON VISTA1(CAR)
   *
ERROR at line 2:
ORA-00942: table or view does not exist 
  \end{verbatim}
  \item{Explicación:} No se puede crear un índice sobre una vista.
\end{itemize}
\end{document}