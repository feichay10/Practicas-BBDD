\documentclass[11pt]{report}

% Paquetes y configuraciones adicionales
\usepackage{graphicx}
\usepackage[export]{adjustbox}
\usepackage{caption}
\usepackage{float}
\usepackage{titlesec}
\usepackage{geometry}
\usepackage[hidelinks]{hyperref}
\usepackage{titling}
\usepackage{titlesec}
\usepackage{parskip}
\usepackage{wasysym}
\usepackage{tikzsymbols}
\usepackage{fancyvrb}
\usepackage{xurl}
\usepackage{hyperref}
\usepackage{subcaption}

\usepackage{listings}
\usepackage{xcolor}

\usepackage[spanish]{babel}

\newcommand{\subtitle}[1]{
  \posttitle{
    \par\end{center}
    \begin{center}\large#1\end{center}
    \vskip0.5em}
}

% Configura los márgenes
\geometry{
  left=2cm,   % Ajusta este valor al margen izquierdo deseado
  right=2cm,  % Ajusta este valor al margen derecho deseado
  top=3cm,
  bottom=3cm,
}

% Configuración de los títulos de las secciones
\titlespacing{\section}{0pt}{\parskip}{\parskip}
\titlespacing{\subsection}{0pt}{\parskip}{\parskip}
\titlespacing{\subsubsection}{0pt}{\parskip}{\parskip}

% Redefinir el formato de los capítulos y añadir un punto después del número
\makeatletter
\renewcommand{\@makechapterhead}[1]{%
  \vspace*{0\p@} % Ajusta este valor para el espaciado deseado antes del título del capítulo
  {\parindent \z@ \raggedright \normalfont
    \ifnum \c@secnumdepth >\m@ne
        \huge\bfseries \thechapter.\ % Añade un punto después del número
    \fi
    \interlinepenalty\@M
    #1\par\nobreak
    \vspace{10pt} % Ajusta este valor para el espacio deseado después del título del capítulo
  }}
\makeatother

% Configura para que cada \chapter no comience en una pagina nueva
\makeatletter
\renewcommand\chapter{\@startsection{chapter}{0}{\z@}%
    {-3.5ex \@plus -1ex \@minus -.2ex}%
    {2.3ex \@plus.2ex}%
    {\normalfont\Large\bfseries}}
\makeatother

% Configurar los colores para el código
\definecolor{codegreen}{rgb}{0,0.6,0}
\definecolor{codegray}{rgb}{0.5,0.5,0.5}
\definecolor{codepurple}{rgb}{0.58,0,0.82}
\definecolor{backcolour}{rgb}{0.95,0.95,0.92}

% Configurar el estilo para el código
\lstdefinestyle{mystyle}{
  backgroundcolor=\color{backcolour},   
  commentstyle=\color{codegreen},
  keywordstyle=\color{magenta},
  numberstyle=\tiny\color{codegray},
  stringstyle=\color{codepurple},
  basicstyle=\ttfamily\footnotesize,
  breakatwhitespace=false,         
  breaklines=true,                 
  captionpos=b,                    
  keepspaces=true,                 
  numbers=left,                    
  numbersep=5pt,                  
  showspaces=false,                
  showstringspaces=false,
  showtabs=false,                  
  tabsize=2
}

\begin{document}

% Portada del informe
\title{Practica 05. Consultas con múltiples tablas}
\subtitle{Bases de Datos}
\author{Cheuk Kelly Ng Pante (alu0101364544@ull.edu.es)}
\date{\today}

\maketitle

\pagestyle{empty} % Desactiva la numeración de página para el índice

% Índice
\tableofcontents

% Nueva página
\cleardoublepage

\pagestyle{plain} % Vuelve a activar la numeración de página
\setcounter{page}{1} % Reinicia el contador de página a 1

% Secciones del informe
% Capitulo 1
% \chapter{Introducción}
% \begin{verbatim}
% SQL> SELECT SYSDATE FROM DUAL;

% SYSDATE                                                                         
% ---------                                                                       
% 28-NOV-23  
% \end{verbatim}

\chapter{Listar los nombres de asignaturas adscritas a áreas cuyo nombre empiece por ‘A’.}
\begin{itemize}
  \item Consulta:
  \begin{verbatim}
SQL> REM 1
SQL> SELECT A
  2  FROM ASIGNATURA WHERE
  3  CAR IN (SELECT CAR FROM AREA WHERE AR LIKE 'A%');
  \end{verbatim}
  \item{Resultado:}
  \begin{verbatim}
A                                                                               
--------------------------------------------------                              
ALGEBRA                                                                         
ANALISIS COMPLEJO                                                               
ASTRONOMIA    
  \end{verbatim}
\end{itemize}

\chapter{Listar los nombres de asignaturas adscritas a áreas cuyo nombre termine en ‘A’.}
\begin{itemize}
  \item Consulta:
  \begin{verbatim}
SQL> REM 2
SQL> SELECT A
  2  FROM ASIGNATURA WHERE
  3  CAR IN (SELECT CAR FROM AREA WHERE AR LIKE '%A');
  \end{verbatim}
  \item{Resultado:}
  \begin{verbatim}
A                                                                               
--------------------------------------------------                              
ALGEBRA                                                                         
ASTRONOMIA                                                                      
DIDACTICA DE LA MATEMATICA                                                      
OPTIMIZACION                                                                    
CALCULO   
  \end{verbatim}
\end{itemize}

\chapter{Listar los nombres de asignaturas que lleven la palabra ‘DATOS’.}% \begin{itemize}
\begin{itemize}
  \item Consulta:
  \begin{verbatim}
SQL> REM 3
SQL> SELECT A
  2  FROM ASIGNATURA
  3  WHERE A LIKE '%DATOS%';
  \end{verbatim}
  \item{Resultado:}
  \begin{verbatim}
A                                                                               
--------------------------------------------------                              
BASES DE DATOS                                                                  
ALMACENES DE DATOS                                                              
MINERIA DE DATOS  
  \end{verbatim}
\end{itemize}

\chapter{Listar los DNI de los profesores en cuyo nombre el tercer carácter sea ‘R’.}
\begin{itemize}
  \item Consulta:
  \begin{verbatim}
SQL> REM 4
SQL> SELECT DNI
  2  FROM PROFESOR
  3  WHERE P LIKE '__R%';
  \end{verbatim}
  \item{Resultado:}
  \begin{verbatim}
       DNI                                                                      
----------                                                                      
      2222                                                                      
      4444                                                                      
      6666                                                                      
      7777  
  \end{verbatim}
\end{itemize}

\chapter{Listar, sin contar duplicados, los DNI de los profesores con nombres de, a lo sumo, 5 caracteres de longitud.}
\begin{itemize}
  \item Consulta:
  \begin{verbatim}
SQL> REM 5
SQL> SELECT UNIQUE P
  2  FROM PROFESOR
  3  WHERE LENGTH(P) <= 5;
  \end{verbatim}
  \item{Resultado:}
  \begin{verbatim}
P                                                                               
------------------------------------------------------------                    
MARIO                                                                           
JUAN                                                                            
PEDRO                                                                           
IVAN                                                                            
DAVID                                                                           
MARIA

6 rows selected.
  \end{verbatim}
\end{itemize}

\chapter{Listar, sin contar duplicados, los DNI de los profesores con nombres de, al menos, 5 caracteres de longitud.}
\begin{itemize}
  \item Consulta:
  \begin{verbatim}
SQL> SELECT UNIQUE P
  2  FROM PROFESOR
  3  WHERE LENGTH(P) >= 5;
  \end{verbatim}
  \item{Resultado:}
  \begin{verbatim}
P                                                                               
------------------------------------------------------------                    
MARIO                                                                           
PEDRO                                                                           
FRANCISCO                                                                       
SOLEDAD                                                                         
CARLOS                                                                          
CARMEN                                                                          
DAVID                                                                           
MARIA                                                                           
JOSE MANUEL                                                                     
ANGELA                                                                          

10 rows selected.
  \end{verbatim}
\end{itemize}

\chapter{Listar los nombres de los profesores que actualmente imparten alguna asignatura.}
\begin{itemize}
  \item Consulta:
  \begin{verbatim}
SQL> REM 7
SQL> SELECT UNIQUE P
  2  FROM PROFESOR NATURAL JOIN PLAN_DOCENTE
  3  WHERE FF IS NULL AND FI < SYSDATE;
  \end{verbatim}
  \item{Resultado:}
  \begin{verbatim}
P                                                                               
------------------------------------------------------------                    
MARIO                                                                           
JUAN                                                                            
PEDRO                                                                           
FRANCISCO                                                                       
SOLEDAD                                                                         
DAVID                                                                           
CARLOS                                                                          
IVAN                                                                            
MARIA                                                                           
JOSE MANUEL                                                                     
ANGELA                                                                          

11 rows selected.
  \end{verbatim}
\end{itemize}

\chapter{Nombres de los profesores que han impartido la asignatura con código 8.}
\begin{itemize}
  \item Consulta:
  \begin{verbatim}
SQL> REM 8
SQL> SELECT P
  2  FROM PROFESOR NATURAL JOIN ASIGNATURA
  3  WHERE CAS = 8;
  \end{verbatim}
  \item{Resultado:}
  \begin{verbatim}
P                                                                               
------------------------------------------------------------                    
JUAN                                                                            
JOSE MANUEL  
  \end{verbatim}
\end{itemize}

\chapter{Listar las t uplas de la tabla PLAN\_DOCENTE ordenadas de forma ascendente, según el campo FF.}
\begin{itemize}
  \item Consulta:
  \begin{verbatim}
SQL> REM 9
SQL> SELECT * FROM PLAN_DOCENTE
  2  ORDER BY FF ASC;
  \end{verbatim}
  \item{Resultado:}
  \begin{verbatim}
DNI        CAS        CTA        CPA        CLA FI        FF             
---------- ---------- ---------- ---------- ---------- --------- ---------      
      2222          3        1.5          0        1.5 01-SEP-06 31-AUG-07      
      1010          2        1.5        1.5          3 01-SEP-05 31-AUG-08      
      1010          9          3          0          3 01-SEP-08 31-AUG-09      
      1111          8          3        1.5        1.5 01-SEP-07 31-AUG-09      
      4444          4        1.5          0        1.5 01-SEP-08 31-AUG-10      
      6666         10          3        1.5        1.5 01-SEP-08 31-AUG-11      
      1010          9        1.5          0        1.5 01-SEP-09                
      9999          7          3          3          0 01-SEP-10                
      5555          6          3          3          0 31-MAR-10                
      8888         11          6          0          0 01-SEP-09                
      2020          3        1.5          0        1.5 01-SEP-08                

        DNI        CAS        CTA        CPA        CLA FI        FF             
---------- ---------- ---------- ---------- ---------- --------- ---------      
      7777         12        4.5          3          0 01-SEP-10                
      3333          9        1.5          0        1.5 01-SEP-09                
      2222          4        1.5          0        1.5 01-SEP-09                
      3030          8          0        1.5        1.5 01-SEP-09                
      1111          8          3          0          0 01-SEP-09                
      4444          5          3          0          0 01-SEP-10                
      4444          1          3        1.5        1.5 01-SEP-11                
      3333          2        1.5        1.5          3 01-SEP-08                

19 rows selected.
  \end{verbatim}
\end{itemize}

\chapter{Listar las t uplas de la tabla PLAN\_DOCENTE ordenadas de forma descendente, según el campo FF.}
\begin{itemize}
  \item Consulta:
  \begin{verbatim}
SQL> REM 10
SQL> SELECT * FROM PLAN_DOCENTE
  2  ORDER BY FF DESC;
  \end{verbatim}
  \item{Resultado:}
  \begin{verbatim}
DNI        CAS        CTA        CPA        CLA FI        FF             
---------- ---------- ---------- ---------- ---------- --------- ---------      
      4444          1          3        1.5        1.5 01-SEP-11                
      3333          2        1.5        1.5          3 01-SEP-08                
      7777         12        4.5          3          0 01-SEP-10                
      2020          3        1.5          0        1.5 01-SEP-08                
      8888         11          6          0          0 01-SEP-09                
      5555          6          3          3          0 31-MAR-10                
      9999          7          3          3          0 01-SEP-10                
      4444          5          3          0          0 01-SEP-10                
      1111          8          3          0          0 01-SEP-09                
      3030          8          0        1.5        1.5 01-SEP-09                
      2222          4        1.5          0        1.5 01-SEP-09                

        DNI        CAS        CTA        CPA        CLA FI        FF             
---------- ---------- ---------- ---------- ---------- --------- ---------      
      3333          9        1.5          0        1.5 01-SEP-09                
      1010          9        1.5          0        1.5 01-SEP-09                
      6666         10          3        1.5        1.5 01-SEP-08 31-AUG-11      
      4444          4        1.5          0        1.5 01-SEP-08 31-AUG-10      
      1010          9          3          0          3 01-SEP-08 31-AUG-09      
      1111          8          3        1.5        1.5 01-SEP-07 31-AUG-09      
      1010          2        1.5        1.5          3 01-SEP-05 31-AUG-08      
      2222          3        1.5          0        1.5 01-SEP-06 31-AUG-07      

19 rows selected.
  \end{verbatim}
\end{itemize}

\chapter{Nombres de los profesores que han impartido la asignatura ‘OPTIMIZACIÓN’ en la titulación GII. Ordena los nombres ascendentemente.}
\begin{itemize}
  \item Consulta:
  \begin{verbatim}
SQL> REM 11
SQL> SELECT P
  2  FROM PROFESOR NATURAL JOIN PLAN_DOCENTE NATURAL JOIN ASIGNATURA
  3  WHERE A = 'OPTIMIZACION' AND T='GII'
  4  ORDER BY P ASC;
  \end{verbatim}
  \item{Resultado:}
  \begin{verbatim}
P                                                                               
------------------------------------------------------------                    
JOSE MANUEL                                                                     
JUAN                                                                            
JUAN  
  \end{verbatim}
\end{itemize}

\chapter{Listar los nombres de los profesores del departamento ‘MATEMÁTICA FUNDAMENTAL’. Ordena los nombres descendentemente.}
\begin{itemize}
  \item Consulta:
  \begin{verbatim}
SQL> REM 12
SQL> SELECT UNIQUE P
  2  FROM PROFESOR NATURAL JOIN AREA NATURAL JOIN DEPARTAMENTO
  3  WHERE D = 'MATEMATICA FUNDAMENTAL'
  4  ORDER BY P ASC;
  \end{verbatim}
  \item{Resultado:}
  \begin{verbatim}
P                                                                               
------------------------------------------------------------                    
IVAN  
  \end{verbatim}
\end{itemize}

\chapter{Listar los nombres de las asignaturas impartidas por el profesor con DNI 1010.}
\begin{itemize}
  \item Consulta:
  \begin{verbatim}
SQL> REM 13
SQL> SELECT UNIQUE A
  2  FROM ASIGNATURA NATURAL JOIN PLAN_DOCENTE
  3  WHERE DNI = 1010;
  \end{verbatim}
  \item{Resultado:}
  \begin{verbatim}
A                                                                               
--------------------------------------------------                              
GESTION DE RIESGOS                                                              
INTELIGENCIA ARTIFICIAL  
  \end{verbatim}
\end{itemize}

\chapter{Listar los nombres de las asignaturas impartidas por el profesor con nombre ‘DAVID’.}
\begin{itemize}
  \item Consulta:
  \begin{verbatim}
SQL> SELECT UNIQUE A
  2  FROM ASIGNATURA NATURAL JOIN PLAN_DOCENTE
  3  NATURAL JOIN PROFESOR
  4  WHERE P = 'DAVID';
  \end{verbatim}
  \item{Resultado:}
  \begin{verbatim}
A                                                                               
--------------------------------------------------                              
GESTION DE RIESGOS                                                              
INTELIGENCIA ARTIFICIAL  
  \end{verbatim}
\end{itemize}

\chapter{Listar los nombres de las áreas adscritas al departamento ‘ESTADISTICA, INVESTIGACIÓN OPERATIVA Y COMPUTACIÓN’.}
\begin{itemize}
  \item Consulta:
  \begin{verbatim}
SQL> REM 15
SQL> SELECT AR
  2  FROM AREA NATURAL JOIN DEPARTAMENTO
  3  WHERE D = 'ESTADISTICA, INVESTIGACION OPERATIVA Y COMPUTACION';
  \end{verbatim}
  \item{Resultado:}
  \begin{verbatim}
AR                                                                              
------------------------------------------------------------                    
CIENCIAS DE LA COMPUTACION E INTELIGENCIA ARTIFICIAL                            
ESTADISTICA E INVESTIGACION OPERATIVA                                           
LENGUAJES Y SISTEMAS INFORMATICOS  
  \end{verbatim}
\end{itemize}

\chapter{Listar los nombres de las asignaturas impartidas actualmente por catedráticos de universidad (categoría CU).}
\begin{itemize}
  \item Consulta:
  \begin{verbatim}
SQL> REM 16
SQL> SELECT A FROM ASIGNATURA NATURAL JOIN PLAN_DOCENTE NATURAL JOIN PROFESOR
  2  WHERE CAT = 'CU';
  \end{verbatim}
  \item{Resultado:}
  \begin{verbatim}
A                                                                               
--------------------------------------------------                              
OPTIMIZACION                                                                    
OPTIMIZACION                                                                    
ALMACENES DE DATOS 
  \end{verbatim}
\end{itemize}

\chapter{Listar los nombres de las asignaturas que siempre han sido impartidas por catedráticos de universidad (categoría CU).}
\begin{itemize}
  \item Consulta:
  \begin{verbatim}
SQL> SELECT UNIQUE A
  2  FROM ASIGNATURA NATURAL JOIN PLAN_DOCENTE NATURAL JOIN PROFESOR
  3  WHERE DNI=ALL(SELECT DNI FROM PROFESOR WHERE CAT='CU');
  \end{verbatim}
  \item{Resultado:}
  \begin{verbatim}
no rows selected
  \end{verbatim}
\end{itemize}

\chapter{Listar los nombres de asignaturas adscritas a ‘LENGUAJES Y SISTEMAS INFORMÁTICOS’ o a ‘CIENCIAS DE LA COMPUTACIÓN E INTELIGENCIA ARTIFICIAL’.}
\begin{itemize}
  \item Consulta:
  \begin{verbatim}
SQL> REM 18
SQL> SELECT A
  2  FROM ASIGNATURA NATURAL JOIN AREA
  3  WHERE AR = 'LENGUAJES Y SISTEMAS INFORMATICOS' 
  4  OR AR = 'CIENCIAS DE LA COMPUTACION E INTELIGENCIA ARTIFICIAL';
  \end{verbatim}
  \item{Resultado:}
  \begin{verbatim}
A                                                                               
--------------------------------------------------                              
GESTION DE RIESGOS                                                              
INTELIGENCIA ARTIFICIAL                                                         
INFORMATICA BASICA                                                              
MINERIA DE DATOS                                                                
ALMACENES DE DATOS                                                              
BASES DE DATOS                                                                  

6 rows selected.
  \end{verbatim}
\end{itemize}

\chapter{Listar los nombres de profesores que actualmente dan clases en las titulaciones ‘GII’ o en ‘MII’.}
\begin{itemize}
  \item Consulta:
  \begin{verbatim}
SQL> rem 19
SQL> SELECT DISTINCT P
  2  FROM PROFESOR NATURAL JOIN ASIGNATURA NATURAL JOIN PLAN_DOCENTE
  3  WHERE (T='GII' OR T='MII') AND FI < SYSDATE AND FF IS NULL;
  \end{verbatim}
  \item{Resultado:}
  \begin{verbatim}
  P                                                                               
  ------------------------------------------------------------                    
  JUAN                                                                            
  PEDRO                                                                           
  SOLEDAD                                                                         
  CARLOS                                                                          
  IVAN                                                                            
  DAVID                                                                           
  MARIA                                                                           
  JOSE MANUEL                                                                     
  ANGELA                                                                          
  
  9 rows selected.                                                                   
  \end{verbatim}
\end{itemize}

\chapter{Listar los nombres de profesores que actualmente dan clases en las titulaciones ‘GII’ y en ‘MII’ simultáneamente.}
\begin{itemize}
  \item Consulta:
  \begin{verbatim}
SQL> REM 20
SQL> SELECT DISTINCT P
  2  FROM PROFESOR NATURAL JOIN ASIGNATURA NATURAL JOIN PLAN_DOCENTE
  3  WHERE T = 'GII' AND T = 'MII' AND FI < SYSDATE AND FF IS NULL;
  \end{verbatim}
  \item{Resultado:}
  \begin{verbatim}
no rows selected
  \end{verbatim}
\end{itemize}

\chapter{Listar los nombres de profesores que actualmente no imparten ninguna asignatura.}
\begin{itemize}
  \item Consulta:
  \begin{verbatim}
SQL> REM 21
SQL> SELECT P
  2  FROM PROFESOR
  3  WHERE DNI NOT IN
  4  (SELECT DNI FROM PLAN_DOCENTE WHERE FF IN NULL);
  \end{verbatim}
  \item{Resultado:}
  \begin{verbatim}
P                                                                               
------------------------------------------------------------                    
DAVID                                                                           
JUAN                                                                            
SOLEDAD                                                                         
CARLOS                                                                          
JOSE MANUEL                                                                     
PEDRO                                                                           
MARIA                                                                           
IVAN                                                                            
CARMEN                                                                          
MARIO                                                                           
FRANCISCO                                                                       

P                                                                               
------------------------------------------------------------                    
ANGELA                                                                          

12 rows selected.
  \end{verbatim}
\end{itemize}

\chapter{Listar los nombres de asignaturas impartidas en la titulación GII.}
\begin{itemize}
  \item Consulta:
  \begin{verbatim}
SQL> REM 22
SQL> SELECT A
  2  FROM ASIGNATURA
  3  WHERE T = 'GII';
  \end{verbatim}
  \item{Resultado:}
  \begin{verbatim}
A                                                                               
--------------------------------------------------                              
BASES DE DATOS                                                                  
INTELIGENCIA ARTIFICIAL                                                         
INFORMATICA BASICA                                                              
ALGEBRA                                                                         
CALCULO                                                                         
OPTIMIZACION                                                                    
GESTION DE RIESGOS                                                              

7 rows selected.
  \end{verbatim}
\end{itemize}

\chapter{Listar los nombres de las áreas de conocimiento y los nombres de las asignaturas que pertenecen a ellos.}
\begin{itemize}
  \item Consulta:
  \begin{verbatim}
SQL> REM 23
SQL> SET LINESIZE 180
SQL> SELECT AR, A
  2  FROM AREA
  3  NATURAL JOIN ASIGNATURA;
  \end{verbatim}
  \item{Resultado:}
  \begin{verbatim}
AR                                                           A                                                                                                                      
------------------------------------------------------------ --------------------------------------------------                                                                     
ALGEBRA                                                      ALGEBRA                                                                                                                
ANALISIS MATEMATICO                                          ANALISIS COMPLEJO                                                                                                      
ASTRONOMIA Y ASTROFISICA                                     ASTRONOMIA                                                                                                             
CIENCIAS DE LA COMPUTACION E INTELIGENCIA ARTIFICIAL         GESTION DE RIESGOS                                                                                                     
CIENCIAS DE LA COMPUTACION E INTELIGENCIA ARTIFICIAL         INTELIGENCIA ARTIFICIAL                                                                                                
DIDACTICA DE LA MATEMATICA                                   DIDACTICA DE LA MATEMATICA                                                                                             
ESTADISTICA E INVESTIGACION OPERATIVA                        OPTIMIZACION                                                                                                           
LENGUAJES Y SISTEMAS INFORMATICOS                            INFORMATICA BASICA                                                                                                     
LENGUAJES Y SISTEMAS INFORMATICOS                            MINERIA DE DATOS                                                                                                       
LENGUAJES Y SISTEMAS INFORMATICOS                            ALMACENES DE DATOS                                                                                                     
LENGUAJES Y SISTEMAS INFORMATICOS                            BASES DE DATOS                                                                                                         

AR                                                           A                                                                                                                      
------------------------------------------------------------ --------------------------------------------------                                                                     
MATEMATICA APLICADA                                          CALCULO                                                                                                                

12 rows selected.
  \end{verbatim}
\end{itemize}

\chapter{Listar los nombres de departamentos y los nombres de las áreas adscritas a ellos. Ambos campos deben estar ordenados de forma alfabética.}
\begin{itemize}
  \item Consulta:
  \begin{verbatim}
SQL> REM 24
SQL> SELECT D, AR
  2  FROM DEPARTAMENTO NATURAL JOIN AREA
  3  ORDER BY D, AR;
  \end{verbatim}
  \item{Resultado:}
  \begin{verbatim}
D                                                            AR                                                                                                                     
------------------------------------------------------------ ------------------------------------------------------------                                                           
ANALISIS MATEMATICO                                          ANALISIS MATEMATICO                                                                                                    
ANALISIS MATEMATICO                                          DIDACTICA DE LA MATEMATICA                                                                                             
ANALISIS MATEMATICO                                          MATEMATICA APLICADA                                                                                                    
ASTROFISICA                                                  ASTRONOMIA Y ASTROFISICA                                                                                               
ESTADISTICA, INVESTIGACION OPERATIVA Y COMPUTACION           CIENCIAS DE LA COMPUTACION E INTELIGENCIA ARTIFICIAL                                                                   
ESTADISTICA, INVESTIGACION OPERATIVA Y COMPUTACION           ESTADISTICA E INVESTIGACION OPERATIVA                                                                                  
ESTADISTICA, INVESTIGACION OPERATIVA Y COMPUTACION           LENGUAJES Y SISTEMAS INFORMATICOS                                                                                      
MATEMATICA FUNDAMENTAL                                       ALGEBRA                                                                                                                

8 rows selected.
  \end{verbatim}
\end{itemize}

\chapter{Listar los nombres de departamentos y los profesores de cada uno de ellos. Ambos campos deben estar ordenados de forma alfabética.}
\begin{itemize}
  \item Consulta:
  \begin{verbatim}
SQL> REM 25
SQL> SELECT D, P
  2  FROM DEPARTAMENTO NATURAL JOIN AREA NATURAL JOIN PROFESOR
  3  ORDER BY D, P;
  \end{verbatim}
  \item{Resultado:}
  \begin{verbatim}
D                                                            P                                                                                                                      
------------------------------------------------------------ ------------------------------------------------------------                                                           
ANALISIS MATEMATICO                                          ANGELA                                                                                                                 
ANALISIS MATEMATICO                                          FRANCISCO                                                                                                              
ANALISIS MATEMATICO                                          MARIO                                                                                                                  
ASTROFISICA                                                  CARMEN                                                                                                                 
ESTADISTICA, INVESTIGACION OPERATIVA Y COMPUTACION           CARLOS                                                                                                                 
ESTADISTICA, INVESTIGACION OPERATIVA Y COMPUTACION           DAVID                                                                                                                  
ESTADISTICA, INVESTIGACION OPERATIVA Y COMPUTACION           JOSE MANUEL                                                                                                            
ESTADISTICA, INVESTIGACION OPERATIVA Y COMPUTACION           JUAN                                                                                                                   
ESTADISTICA, INVESTIGACION OPERATIVA Y COMPUTACION           MARIA                                                                                                                  
ESTADISTICA, INVESTIGACION OPERATIVA Y COMPUTACION           PEDRO                                                                                                                  
ESTADISTICA, INVESTIGACION OPERATIVA Y COMPUTACION           SOLEDAD                                                                                                                

D                                                            P                                                                                                                      
------------------------------------------------------------ ------------------------------------------------------------                                                           
MATEMATICA FUNDAMENTAL                                       IVAN                                                                                                                   

12 rows selected.
  \end{verbatim}
\end{itemize}

\end{document}