\documentclass[11pt]{report}

% Paquetes y configuraciones adicionales
\usepackage{graphicx}
\usepackage[export]{adjustbox}
\usepackage{caption}
\usepackage{float}
\usepackage{titlesec}
\usepackage{geometry}
\usepackage[hidelinks]{hyperref}
\usepackage{titling}
\usepackage{titlesec}
\usepackage{parskip}
\usepackage{wasysym}
\usepackage{tikzsymbols}
\usepackage{fancyvrb}
\usepackage{xurl}
\usepackage{hyperref}
\usepackage{subcaption}

\usepackage{listings}
\usepackage{xcolor}

\usepackage[spanish]{babel}

\newcommand{\subtitle}[1]{
  \posttitle{
    \par\end{center}
    \begin{center}\large#1\end{center}
    \vskip0.5em}
}

% Configura los márgenes
\geometry{
  left=2cm,   % Ajusta este valor al margen izquierdo deseado
  right=2cm,  % Ajusta este valor al margen derecho deseado
  top=3cm,
  bottom=3cm,
}

% Configuración de los títulos de las secciones
\titlespacing{\section}{0pt}{\parskip}{\parskip}
\titlespacing{\subsection}{0pt}{\parskip}{\parskip}
\titlespacing{\subsubsection}{0pt}{\parskip}{\parskip}

% Redefinir el formato de los capítulos y añadir un punto después del número
\makeatletter
\renewcommand{\@makechapterhead}[1]{%
  \vspace*{0\p@} % Ajusta este valor para el espaciado deseado antes del título del capítulo
  {\parindent \z@ \raggedright \normalfont
    \ifnum \c@secnumdepth >\m@ne
        \huge\bfseries \thechapter.\ % Añade un punto después del número
    \fi
    \interlinepenalty\@M
    #1\par\nobreak
    \vspace{10pt} % Ajusta este valor para el espacio deseado después del título del capítulo
  }}
\makeatother

% Configura para que cada \chapter no comience en una pagina nueva
\makeatletter
\renewcommand\chapter{\@startsection{chapter}{0}{\z@}%
    {-3.5ex \@plus -1ex \@minus -.2ex}%
    {2.3ex \@plus.2ex}%
    {\normalfont\Large\bfseries}}
\makeatother

% Configurar los colores para el código
\definecolor{codegreen}{rgb}{0,0.6,0}
\definecolor{codegray}{rgb}{0.5,0.5,0.5}
\definecolor{codepurple}{rgb}{0.58,0,0.82}
\definecolor{backcolour}{rgb}{0.95,0.95,0.92}

% Configurar el estilo para el código
\lstdefinestyle{mystyle}{
  backgroundcolor=\color{backcolour},   
  commentstyle=\color{codegreen},
  keywordstyle=\color{magenta},
  numberstyle=\tiny\color{codegray},
  stringstyle=\color{codepurple},
  basicstyle=\ttfamily\footnotesize,
  breakatwhitespace=false,         
  breaklines=true,                 
  captionpos=b,                    
  keepspaces=true,                 
  numbers=left,                    
  numbersep=5pt,                  
  showspaces=false,                
  showstringspaces=false,
  showtabs=false,                  
  tabsize=2
}

\begin{document}

% Portada del informe
\title{Practica 07. Control de Transacciones}
\subtitle{Bases de Datos}
\author{Cheuk Kelly Ng Pante (alu0101364544@ull.edu.es)}
\date{\today}

\maketitle

\pagestyle{empty} % Desactiva la numeración de página para el índice

% Índice
\tableofcontents

% Nueva página
\cleardoublepage

\pagestyle{plain} % Vuelve a activar la numeración de página
\setcounter{page}{1} % Reinicia el contador de página a 1

% Secciones del informe
% Pregunta 1
\chapter{Listar todas las t-uplas de la tabla PROFESOR.}
\begin{itemize}
  \item Consulta:
  \begin{verbatim}
SQL> SELECT * FROM PROFESOR;
  \end{verbatim}
  \item{Resultado:}
  \begin{verbatim}
       DNI P                                                                   CAR CAT              
---------- ------------------------------------------------------------ ---------- -----            
      1111 JUAN                                                                  6 CU               
      2222 CARLOS                                                                7 TU               
      3333 PEDRO                                                                 4 TEU              
      4444 MARIA                                                                 7 TU               
      5555 IVAN                                                                  1 CEU              
      6666 CARMEN                                                                3 CD               
      7777 MARIO                                                                 2 TU               
      8888 FRANCISCO                                                             5 TU               
      9999 ANGELA                                                                8 TEU              
      1010 DAVID                                                                 4 TU               
      2020 SOLEDAD                                                               7 CU               
      3030 JOSE MANUEL                                                           6 TEU              

12 rows selected.
  \end{verbatim}
\end{itemize}

% Pregunta 2
\chapter{Añadir la siguiente t-upla a la tabla PROFESOR: (4040, ‘CARMELO’, 7, TU)}
\begin{itemize}
  \item Consulta:
  \begin{verbatim}
SQL> INSERT INTO PROFESOR
  2  VALUES(4040, 'CARMELO', 7, 'TU');
  \end{verbatim}
  \item{Resultado:}
  \begin{verbatim}
1 row created.
  \end{verbatim}
\end{itemize}

\newpage

% Pregunta 3
\chapter{Listar todas las t-uplas de la tabla PROFESOR.}
\begin{itemize}
  \item Consulta:
  \begin{verbatim}
SQL> SELECT * FROM PROFESOR;
  \end{verbatim}
  \item{Resultado:}
  \begin{verbatim}
       DNI P                                                                   CAR CAT              
---------- ------------------------------------------------------------ ---------- -----            
      4040 CARMELO                                                               7 TU               
      1111 JUAN                                                                  6 CU               
      2222 CARLOS                                                                7 TU               
      3333 PEDRO                                                                 4 TEU              
      4444 MARIA                                                                 7 TU               
      5555 IVAN                                                                  1 CEU              
      6666 CARMEN                                                                3 CD               
      7777 MARIO                                                                 2 TU               
      8888 FRANCISCO                                                             5 TU               
      9999 ANGELA                                                                8 TEU              
      1010 DAVID                                                                 4 TU               
      2020 SOLEDAD                                                               7 CU               
      3030 JOSE MANUEL                                                           6 TEU              

13 rows selected.
  \end{verbatim}
\end{itemize}

% Pregunta 4
\chapter{Deshacer los cambios.}
\begin{itemize}
  \item Consulta:
  \begin{verbatim}
SQL> ROLLBACK WORK;
  \end{verbatim}
  \item{Resultado:}
  \begin{verbatim}
Rollback complete.
  \end{verbatim}
\end{itemize}

\newpage

% Pregunta 5
\chapter{Listar todas las t-uplas de la tabla PROFESOR.}
\begin{itemize}
  \item Consulta:
  \begin{verbatim}
SQL> REM 5
SQL> SELECT * FROM PROFESOR;
  \end{verbatim}
  \item{Resultado:}
  \begin{verbatim}
       DNI P                                                                   CAR CAT              
---------- ------------------------------------------------------------ ---------- -----            
      1111 JUAN                                                                  6 CU               
      2222 CARLOS                                                                7 TU               
      3333 PEDRO                                                                 4 TEU              
      4444 MARIA                                                                 7 TU               
      5555 IVAN                                                                  1 CEU              
      6666 CARMEN                                                                3 CD               
      7777 MARIO                                                                 2 TU               
      8888 FRANCISCO                                                             5 TU               
      9999 ANGELA                                                                8 TEU              
      1010 DAVID                                                                 4 TU               
      2020 SOLEDAD                                                               7 CU               
      3030 JOSE MANUEL                                                           6 TEU              

12 rows selected.
  \end{verbatim}
\end{itemize}

% Pregunta 6
\chapter{Añadir la siguiente t-upla a la tabla PROFESOR: (4040, ‘CARMELO’, 7, TU).}
\begin{itemize}
  \item Consulta:
  \begin{verbatim}
SQL> INSERT INTO PROFESOR
  2  VALUES(4040,'CARMELO',7,'TU');
  \end{verbatim}
  \item{Resultado:}
  \begin{verbatim}
1 row created.
  \end{verbatim}
\end{itemize}

\newpage

% Pregunta 7
\chapter{Hacer permanentes los cambios.}
\begin{itemize}
  \item Consulta:
  \begin{verbatim}
SQL> COMMIT WORK;
  \end{verbatim}
  \item{Resultado:}
  \begin{verbatim}
Commit complete.
  \end{verbatim}
\end{itemize}

% Pregunta 8
\chapter{Añadir la siguiente t-upla a la tabla PROFESOR: (5050, ‘CARINA’, 9, CEU). Explica qué ocurre.}
\begin{itemize}
  \item Consulta:
  \begin{verbatim}
SQL> INSERT INTO PROFESOR
  2  VALUES(5050,'CARINA',9,'CEU');
  \end{verbatim}
  \item{Resultado:}
  \begin{verbatim}
INSERT INTO PROFESOR
*
ERROR at line 1:
ORA-02291: integrity constraint (ALU0101364544.SYS_C00410389) violated - parent 
key not found 
  \end{verbatim}
  \item Explicación: La clave foránea no se encuentra en la tabla correspondiente. Es decir, el valor 9 en la columna CAR no tiene un registro que se corresponda en la tabla AREA.
\end{itemize}

% Pregunta 9
\chapter{Añadir la siguiente t-upla a la tabla PROFESOR: (5050, ‘CARINA’, 8, CEU).}
\begin{itemize}
  \item Consulta:
  \begin{verbatim}
SQL> INSERT INTO PROFESOR
  2  VALUES(5050,'CARINA',8,'CEU');
  \end{verbatim}
  \item{Resultado:}
  \begin{verbatim}
1 row created.
  \end{verbatim}
\end{itemize}

\newpage

% Pregunta 10
\chapter{Listar todas las t-uplas de la tabla PROFESOR.}
\begin{itemize}
  \item Consulta:
  \begin{verbatim}
SQL> SELECT * FROM PROFESOR;
  \end{verbatim}
  \item{Resultado:}
  \begin{verbatim}
       DNI P                                                                   CAR CAT              
---------- ------------------------------------------------------------ ---------- -----            
      4040 CARMELO                                                               7 TU               
      5050 CARINA                                                                8 CEU              
      1111 JUAN                                                                  6 CU               
      2222 CARLOS                                                                7 TU               
      3333 PEDRO                                                                 4 TEU              
      4444 MARIA                                                                 7 TU               
      5555 IVAN                                                                  1 CEU              
      6666 CARMEN                                                                3 CD               
      7777 MARIO                                                                 2 TU               
      8888 FRANCISCO                                                             5 TU               
      9999 ANGELA                                                                8 TEU              
      1010 DAVID                                                                 4 TU               
      2020 SOLEDAD                                                               7 CU               
      3030 JOSE MANUEL                                                           6 TEU              

14 rows selected.
  \end{verbatim}
\end{itemize}

% Pregunta 11
\chapter{Abrir una nueva sesión Unix (sesión 2) ¡sin cerrar la original (sesión 1)! Ejecutar de nuevo el SQL*Plus para conectarte a tu base de datos. (¡No olvides hacer un nuevo spool!)}
Realizado

\newpage

% Pregunta 12
\chapter{(Sesión 2) Listar todas las t-uplas de la tabla PROFESOR. Explica qué ocurre.}
\begin{itemize}
  \item Consulta:
  \begin{verbatim}
SQL> SELECT * FROM PROFESOR;
  \end{verbatim}
  \item{Resultado:}
  \begin{verbatim}
       DNI P                                                                   CAR CAT                                                                                                                  
---------- ------------------------------------------------------------ ---------- -----                                                                                                                
      4040 CARMELO                                                               7 TU                                                                                                                   
      1111 JUAN                                                                  6 CU                                                                                                                   
      2222 CARLOS                                                                7 TU                                                                                                                   
      3333 PEDRO                                                                 4 TEU                                                                                                                  
      4444 MARIA                                                                 7 TU                                                                                                                   
      5555 IVAN                                                                  1 CEU                                                                                                                  
      6666 CARMEN                                                                3 CD                                                                                                                   
      7777 MARIO                                                                 2 TU                                                                                                                   
      8888 FRANCISCO                                                             5 TU                                                                                                                   
      9999 ANGELA                                                                8 TEU                                                                                                                  
      1010 DAVID                                                                 4 TU                                                                                                                   
      2020 SOLEDAD                                                               7 CU                                                                                                                   
      3030 JOSE MANUEL                                                           6 TEU                                                                                                                  

13 rows selected.
  \end{verbatim}

  \item{Explicación:} Al no haber hecho los cambios permanentes, no se ha guardado aun la última fila añadida, y es por ello que no aparece.
\end{itemize}

% Pregunta 13
\chapter{(Sesión 2) Añadir la siguiente t-upla a la tabla PROFESOR: (5050, ‘CARINA’, 8, CEU). Explica qué ocurre.}
\begin{itemize}
  \item Consulta:
  \begin{verbatim}
SQL> INSERT INTO PROFESOR
  2  VALUES(5050, 'CARINA', 8, 'CEU');
  \end{verbatim}
  \item{Resultado:}
  \begin{verbatim}
INSERT INTO PROFESOR
*
ERROR at line 1:
ORA-00001: unique constraint (ALU0101364544.SYS_C00475445) violated  
  \end{verbatim}
  \item Explicación: En la sesion 2 se queda en espera porque la sesion 1 ha modificado una linea pero no ha confirmado
\end{itemize}

% Pregunta 14
\chapter{(Sesión 1) Hacer permanentes los cambios.}
\begin{itemize}
  \item Consulta:
  \begin{verbatim}
SQL> COMMIT WORK;
  \end{verbatim}
  \item{Resultado:}
  \begin{verbatim}
Commit complete.
  \end{verbatim}
\end{itemize}

% Pregunta 15
\chapter{(Sesión 2) Listar todas las t-uplas de la tabla PROFESOR. Explica qué ocurre.}
\begin{itemize}
  \item Consulta:
  \begin{verbatim}
SQL> SELECT * FROM PROFESOR;
  \end{verbatim}
  \item{Resultado:}
  \begin{verbatim}
       DNI P                                                                   CAR CAT                                                                                                                  
---------- ------------------------------------------------------------ ---------- -----                                                                                                                
      4040 CARMELO                                                               7 TU                                                                                                                   
      5050 CARINA                                                                8 CEU                                                                                                                  
      1111 JUAN                                                                  6 CU                                                                                                                   
      2222 CARLOS                                                                7 TU                                                                                                                   
      3333 PEDRO                                                                 4 TEU                                                                                                                  
      4444 MARIA                                                                 7 TU                                                                                                                   
      5555 IVAN                                                                  1 CEU                                                                                                                  
      6666 CARMEN                                                                3 CD                                                                                                                   
      7777 MARIO                                                                 2 TU                                                                                                                   
      8888 FRANCISCO                                                             5 TU                                                                                                                   
      9999 ANGELA                                                                8 TEU                                                                                                                  
      1010 DAVID                                                                 4 TU                                                                                                                   
      2020 SOLEDAD                                                               7 CU                                                                                                                   
      3030 JOSE MANUEL                                                           6 TEU                                                                                                                  

14 rows selected.
  \end{verbatim}
  \item Explicación: Al hacer los cambios permanentes, se ha añadido la tupla
\end{itemize}

% Pregunta 16
\chapter{(Sesión 2) Cerrar esta sesión de trabajo y volver a la sesión 1.}
Realizado

% Pregunta 17
\chapter{Eliminar de la tabla ‘PLAN\_DOCENTE’ todas las t-uplas correspondientes al profesor con DNI 4444.}
\begin{itemize}
  \item Consulta:
  \begin{verbatim}
SQL> DELETE FROM PLAN_DOCENTE
  2  WHERE DNI=4444;
  \end{verbatim}
  \item{Resultado:}
  \begin{verbatim}
3 rows deleted.
  \end{verbatim}
\end{itemize}

% Pregunta 18
\chapter{Listar las t-uplas de la tabla ‘PLAN\_DOCENTE’ correspondientes al profesor con DNI 4444.}
\begin{itemize}
  \item Consulta:
  \begin{verbatim}
SQL> SELECT *
  2  FROM PLAN_DOCENTE
  3  WHERE DNI = 4444;
  \end{verbatim}
  \item{Resultado:}
  \begin{verbatim}
no rows selected
  \end{verbatim}
\end{itemize}

% Pregunta 19
\chapter{Establecer un punto de control con el nombre ‘P1’.}
\begin{itemize}
  \item Consulta:
  \begin{verbatim}
SQL> SAVEPOINT P1;
  \end{verbatim}
  \item{Resultado:}
  \begin{verbatim}
Savepoint created.
  \end{verbatim}
\end{itemize}

% Pregunta 20
\chapter{Eliminar de la tabla ‘PLAN\_DOCENTE’ todas las t-uplas correspondientes al profesor con DNI 1010.}
\begin{itemize}
  \item Consulta:
  \begin{verbatim}
SQL> DELETE FROM PLAN_DOCENTE
  2  WHERE DNI=1010;
  \end{verbatim}
  \item{Resultado:}
  \begin{verbatim}
3 rows deleted.
  \end{verbatim}
\end{itemize}

% Pregunta 21
\chapter{Establecer un punto de control con el nombre ‘P2’.}
\begin{itemize}
  \item Consulta:
  \begin{verbatim}
SQL> SAVEPOINT P2;
  \end{verbatim}
  \item{Resultado:}
  \begin{verbatim}
Savepoint created.
  \end{verbatim}
\end{itemize}

% Pregunta 22
\chapter{Eliminar de la tabla ‘PLAN\_DOCENTE’ todas las t-uplas correspondientes al profesor con DNI 2222.}
\begin{itemize}
  \item Consulta:
  \begin{verbatim}
SQL> DELETE FROM PLAN_DOCENTE
  2  WHERE DNI=2222;
  \end{verbatim}
  \item{Resultado:}
  \begin{verbatim}
2 rows deleted.
  \end{verbatim}
\end{itemize}

\newpage

% Pregunta 23
\chapter{Listar las t-uplas de la tabla ‘PLAN\_DOCENTE’.}
\begin{itemize}
  \item Consulta:
  \begin{verbatim}
SQL> SELECT * FROM PLAN_DOCENTE;
  \end{verbatim}
  \item{Resultado:}
  \begin{verbatim}
       DNI        CAS        CTA        CPA        CLA FI        FF                                                                                                                                     
---------- ---------- ---------- ---------- ---------- --------- ---------                                                                                                                              
      1111          8          3        1.5        1.5 01-SEP-07 31-AUG-09                                                                                                                              
      1111          8          3          0          0 01-SEP-09                                                                                                                                        
      3030          8          0        1.5        1.5 01-SEP-09                                                                                                                                        
      3333          2        1.5        1.5          3 01-SEP-08                                                                                                                                        
      9999          7          3          3          0 01-SEP-10                                                                                                                                        
      5555          6          3          3          0 31-MAR-10                                                                                                                                        
      6666         10          3        1.5        1.5 01-SEP-08 31-AUG-11                                                                                                                              
      8888         11          6          0          0 01-SEP-09                                                                                                                                        
      2020          3        1.5          0        1.5 01-SEP-08                                                                                                                                        
      7777         12        4.5          3          0 01-SEP-10                                                                                                                                        
      3333          9        1.5          0        1.5 01-SEP-09                                                                                                                                        

11 rows selected.
  \end{verbatim}
\end{itemize}

% Pregunta 24
\chapter{Añadir a la tabla ‘PLAN\_DOCENTE’ la t-upla (1010, 9, 1.5, 0, 1.5, ‘01-SEP-09’, NULL). Explica qué ocurre.}
\begin{itemize}
  \item Consulta:
  \begin{verbatim}
SQL> INSERT INTO PLAN_DOCENTE
  2  VALUES(1010, 9, 1.5, 0, 1.5, '01-SEP-09', NULL);
  \end{verbatim}
  \item{Resultado:}
  \begin{verbatim}
1 row created.
  \end{verbatim}
\end{itemize}

\newpage

% Pregunta 25
\chapter{Listar las t-uplas de la tabla ‘PLAN\_DOCENTE’.}
\begin{itemize}
  \item Consulta:
  \begin{verbatim}
SQL> SELECT * FROM PLAN_DOCENTE;
  \end{verbatim}
  \item{Resultado:}
  \begin{verbatim}
       DNI        CAS        CTA        CPA        CLA FI        FF                                 
---------- ---------- ---------- ---------- ---------- --------- ---------                          
      1010          9        1.5          0        1.5 01-SEP-09                                    
      1111          8          3        1.5        1.5 01-SEP-07 31-AUG-09                          
      1111          8          3          0          0 01-SEP-09                                    
      3030          8          0        1.5        1.5 01-SEP-09                                    
      3333          2        1.5        1.5          3 01-SEP-08                                    
      9999          7          3          3          0 01-SEP-10                                    
      5555          6          3          3          0 31-MAR-10                                    
      6666         10          3        1.5        1.5 01-SEP-08 31-AUG-11                          
      8888         11          6          0          0 01-SEP-09                                    
      2020          3        1.5          0        1.5 01-SEP-08                                    
      7777         12        4.5          3          0 01-SEP-10                                    
      3333          9        1.5          0        1.5 01-SEP-09                                    

12 rows selected. 
  \end{verbatim}
\end{itemize}

% Pregunta 26
\chapter{Deshacer los cambios hasta el punto de control P2.}
\begin{itemize}
  \item Consulta:
  \begin{verbatim}
SQL> ROLLBACK TO P2;
  \end{verbatim}
  \item{Resultado:}
  \begin{verbatim}
Rollback complete.
  \end{verbatim}
\end{itemize}

% Pregunta 27
\chapter{Listar las t-uplas de la tabla ‘PLAN\_DOCENTE’.}
\begin{itemize}
  \item Consulta:
  \begin{verbatim}
SQL> REM 27
SQL> SELECT * FROM PLAN_DOCENTE;
  \end{verbatim}

  \newpage

  \item{Resultado:}
  \begin{verbatim}
       DNI        CAS        CTA        CPA        CLA FI        FF                                 
---------- ---------- ---------- ---------- ---------- --------- ---------                          
      1111          8          3        1.5        1.5 01-SEP-07 31-AUG-09                          
      1111          8          3          0          0 01-SEP-09                                    
      3030          8          0        1.5        1.5 01-SEP-09                                    
      2222          4        1.5          0        1.5 01-SEP-09                                    
      2222          3        1.5          0        1.5 01-SEP-06 31-AUG-07                          
      3333          2        1.5        1.5          3 01-SEP-08                                    
      9999          7          3          3          0 01-SEP-10                                    
      5555          6          3          3          0 31-MAR-10                                    
      6666         10          3        1.5        1.5 01-SEP-08 31-AUG-11                          
      8888         11          6          0          0 01-SEP-09                                    
      2020          3        1.5          0        1.5 01-SEP-08                                    
      7777         12        4.5          3          0 01-SEP-10                                    
      3333          9        1.5          0        1.5 01-SEP-09                                    

13 rows selected.
  \end{verbatim}
\end{itemize}

% Pregunta 28
\chapter{Deshacer los cambios hasta el punto de control P1.}
\begin{itemize}
  \item Consulta:
  \begin{verbatim}
SQL> ROLLBACK TO P1;
  \end{verbatim}
  \item{Resultado:}
  \begin{verbatim}
Rollback complete.
  \end{verbatim}
\end{itemize}

% Pregunta 29
\chapter{Listar las t-uplas de la tabla ‘PLAN\_DOCENTE’.}
\begin{itemize}
  \item Consulta:
  \begin{verbatim}
SQL> SELECT * FROM PLAN_DOCENTE;
  \end{verbatim}

  \newpage

  \item{Resultado:}
  \begin{verbatim}
       DNI        CAS        CTA        CPA        CLA FI        FF                                 
---------- ---------- ---------- ---------- ---------- --------- ---------                          
      1111          8          3        1.5        1.5 01-SEP-07 31-AUG-09                          
      1111          8          3          0          0 01-SEP-09                                    
      3030          8          0        1.5        1.5 01-SEP-09                                    
      2222          4        1.5          0        1.5 01-SEP-09                                    
      2222          3        1.5          0        1.5 01-SEP-06 31-AUG-07                          
      1010          2        1.5        1.5          3 01-SEP-05 31-AUG-08                          
      3333          2        1.5        1.5          3 01-SEP-08                                    
      1010          9          3          0          3 01-SEP-08 31-AUG-09                          
      1010          9        1.5          0        1.5 01-SEP-09                                    
      9999          7          3          3          0 01-SEP-10                                    
      5555          6          3          3          0 31-MAR-10                                    
      6666         10          3        1.5        1.5 01-SEP-08 31-AUG-11                          
      8888         11          6          0          0 01-SEP-09                                    
      2020          3        1.5          0        1.5 01-SEP-08                                    
      7777         12        4.5          3          0 01-SEP-10                                    
      3333          9        1.5          0        1.5 01-SEP-09                                    

16 rows selected.
  \end{verbatim}
\end{itemize}

% Pregunta 30
\chapter{Deshacer los cambios hasta el inicio de la transacción.}
\begin{itemize}
  \item Consulta:
  \begin{verbatim}
SQL> ROLLBACK;
  \end{verbatim}
  \item{Resultado:}
  \begin{verbatim}
Rollback complete.
  \end{verbatim}
\end{itemize}

% Pregunta 31
\chapter{Listar las t-uplas de la tabla ‘PLAN\_DOCENTE’.}
\begin{itemize}
  \item Consulta:
  \begin{verbatim}
SQL> SELECT * FROM PLAN_DOCENTE;
  \end{verbatim}

  \newpage

  \item{Resultado:}
  \begin{verbatim}
       DNI        CAS        CTA        CPA        CLA FI        FF                                 
---------- ---------- ---------- ---------- ---------- --------- ---------                          
      4444          1          3        1.5        1.5 01-SEP-11                                    
      4444          4        1.5          0        1.5 01-SEP-08 31-AUG-10                          
      4444          5          3          0          0 01-SEP-10                                    
      1111          8          3        1.5        1.5 01-SEP-07 31-AUG-09                          
      1111          8          3          0          0 01-SEP-09                                    
      3030          8          0        1.5        1.5 01-SEP-09                                    
      2222          4        1.5          0        1.5 01-SEP-09                                    
      2222          3        1.5          0        1.5 01-SEP-06 31-AUG-07                          
      1010          2        1.5        1.5          3 01-SEP-05 31-AUG-08                          
      3333          2        1.5        1.5          3 01-SEP-08                                    
      1010          9          3          0          3 01-SEP-08 31-AUG-09                          
      1010          9        1.5          0        1.5 01-SEP-09                                    
      9999          7          3          3          0 01-SEP-10                                    
      5555          6          3          3          0 31-MAR-10                                    
      6666         10          3        1.5        1.5 01-SEP-08 31-AUG-11                          
      8888         11          6          0          0 01-SEP-09                                    
      2020          3        1.5          0        1.5 01-SEP-08                                    
      7777         12        4.5          3          0 01-SEP-10                                    
      3333          9        1.5          0        1.5 01-SEP-09                                    

19 rows selected.
  \end{verbatim}
\end{itemize}


\end{document}