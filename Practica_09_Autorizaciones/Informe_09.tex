\documentclass[11pt]{report}

% Paquetes y configuraciones adicionales
\usepackage{graphicx}
\usepackage[export]{adjustbox}
\usepackage{caption}
\usepackage{float}
\usepackage{titlesec}
\usepackage{geometry}
\usepackage[hidelinks]{hyperref}
\usepackage{titling}
\usepackage{titlesec}
\usepackage{parskip}
\usepackage{wasysym}
\usepackage{tikzsymbols}
\usepackage{fancyvrb}
\usepackage{xurl}
\usepackage{hyperref}
\usepackage{subcaption}

\usepackage{listings}
\usepackage{xcolor}

\usepackage[spanish]{babel}

\newcommand{\subtitle}[1]{
  \posttitle{
    \par\end{center}
    \begin{center}\large#1\end{center}
    \vskip0.5em}
}

% Configura los márgenes
\geometry{
  left=2cm,   % Ajusta este valor al margen izquierdo deseado
  right=2cm,  % Ajusta este valor al margen derecho deseado
  top=3cm,
  bottom=3cm,
}

% Configuración de los títulos de las secciones
\titlespacing{\section}{0pt}{\parskip}{\parskip}
\titlespacing{\subsection}{0pt}{\parskip}{\parskip}
\titlespacing{\subsubsection}{0pt}{\parskip}{\parskip}

% Redefinir el formato de los capítulos y añadir un punto después del número
\makeatletter
\renewcommand{\@makechapterhead}[1]{%
  \vspace*{0\p@} % Ajusta este valor para el espaciado deseado antes del título del capítulo
  {\parindent \z@ \raggedright \normalfont
    \ifnum \c@secnumdepth >\m@ne
        \huge\bfseries \thechapter.\ % Añade un punto después del número
    \fi
    \interlinepenalty\@M
    #1\par\nobreak
    \vspace{10pt} % Ajusta este valor para el espacio deseado después del título del capítulo
  }}
\makeatother

% Configura para que cada \chapter no comience en una pagina nueva
\makeatletter
\renewcommand\chapter{\@startsection{chapter}{0}{\z@}%
    {-3.5ex \@plus -1ex \@minus -.2ex}%
    {2.3ex \@plus.2ex}%
    {\normalfont\Large\bfseries}}
\makeatother

% Configurar los colores para el código
\definecolor{codegreen}{rgb}{0,0.6,0}
\definecolor{codegray}{rgb}{0.5,0.5,0.5}
\definecolor{codepurple}{rgb}{0.58,0,0.82}
\definecolor{backcolour}{rgb}{0.95,0.95,0.92}

% Configurar el estilo para el código
\lstdefinestyle{mystyle}{
  backgroundcolor=\color{backcolour},   
  commentstyle=\color{codegreen},
  keywordstyle=\color{magenta},
  numberstyle=\tiny\color{codegray},
  stringstyle=\color{codepurple},
  basicstyle=\ttfamily\footnotesize,
  breakatwhitespace=false,         
  breaklines=true,                 
  captionpos=b,                    
  keepspaces=true,                 
  numbers=left,                    
  numbersep=5pt,                  
  showspaces=false,                
  showstringspaces=false,
  showtabs=false,                  
  tabsize=2
}

\begin{document}

% Portada del informe
\title{Practica 09. Autorizaciones}
\subtitle{Bases de Datos}
\author{Cheuk Kelly Ng Pante (alu0101364544@ull.edu.es)}
\date{\today}

\maketitle

\pagestyle{empty} % Desactiva la numeración de página para el índice

% Índice
\tableofcontents

% Nueva página
\cleardoublepage

\pagestyle{plain} % Vuelve a activar la numeración de página
\setcounter{page}{1} % Reinicia el contador de página a 1

% Secciones del informe
%Pregunta 1:
\chapter{Permitir al usuario alumno el poder consultas las tablas DEPARTAMENTO, AREA, ASIGNATURA, y TITULACION}
\begin{itemize}
  \item Consulta:
  \begin{verbatim}
SQL> GRANT SELECT
  2  ON DEPARTAMENTO
  3  TO ALUMNO;
  
SQL> GRANT SELECT
  2  ON AREA
  3  TO ALUMNO;
  
SQL> GRANT SELECT
  2  ON ASIGNATURA
  3  TO ALUMNO;
  
SQL> GRANT SELECT
  2  ON TITULACION
  3  TO ALUMNO;
  \end{verbatim}
  \item{Resultado:}
  \begin{verbatim}
Grant succeeded.

Grant succeeded.

Grant succeeded.

Grant succeeded.
  \end{verbatim}
\end{itemize}

%Pregunta 2:
\chapter{Permitir al usuario alumno el poder insertar, modificar y borrar en las tablas DEPARTAMENTO y AREA y modificar en las tablas ASIGNATURA y TITULACION.}
\begin{itemize}
  \item Consulta:
  \begin{verbatim}
SQL> GRANT INSERT, UPDATE, DELETE
  2  ON DEPARTAMENTO
  3  TO ALUMNO;


SQL> GRANT INSERT, UPDATE, DELETE
  2  ON AREA
  3  TO ALUMNO;

SQL> GRANT UPDATE
  2  ON ASIGNATURA
  3  TO ALUMNO;

SQL> GRANT UPDATE
  2  ON TITULACION
  3  TO ALUMNO;
  \end{verbatim}
  \item{Resultado:}
  \begin{verbatim}
Grant succeeded.

Grant succeeded.

Grant succeeded.

Grant succeeded.
  \end{verbatim}
\end{itemize}

%Pregunta 3:
\chapter{Abrir una nueva sesión unix (sesión2) ¡sin cerrar la original (sesión1)!. Ejecutar de nuevo el
SQL*Pllus para conectarte a tu base de datos como el usuario alumno con pasword xxxxx.
(¡No olvides hacer un nuevo spool!)}
Realizado.

\newpage

%Pregunta 4:
\chapter{Listar todas las t-uplas de la tabla DEPARTAMENTO}
\begin{itemize}
  \item Consulta:
  \begin{verbatim}
SQL> SELECT * FROM ALU0101364544.DEPARTAMENTO;
  \end{verbatim}
  \item{Resultado:}
  \begin{verbatim}
        CD D                                                                    
---------- ------------------------------------------------------------         
         1 ANALISIS MATEMATICO                                                  
         2 ASTROFISICA                                                          
         3 ESTADISTICA, INVESTIGACION OPERATIVA Y COMPUTACION                   
         4 MATEMATICA FUNDAMENTAL       
  \end{verbatim}
\end{itemize}

%Pregunta 5:
\chapter{Listar todas las t-uplas de la tabla PROFESOR}
\begin{itemize}
  \item Consulta:
  \begin{verbatim}
SQL> SELECT * FROM ALU0101364544.PROFESOR;
  \end{verbatim}
  \item{Resultado:}
  \begin{verbatim}
SELECT * FROM ALU0101364544.PROFESOR
                            *
ERROR at line 1:
ORA-00942: table or view does not exist 
  \end{verbatim}
\end{itemize}

%Pregunta 6:
\chapter{Insertar la t-upla (5, ‘ECONOMÍA’) en la tabla DEPARTAMENTO}
\begin{itemize}
  \item Consulta:
  \begin{verbatim}
SQL> INSERT INTO ALU0101364544.DEPARTAMENTO
  2  VALUES(5, 'ECONOMIA');
  \end{verbatim}
  \item{Resultado:}
  \begin{verbatim}
1 row created.
  \end{verbatim}
\end{itemize}

%Pregunta 7:
\chapter{Modificar el nombre del departamento con código 5 al valor ‘ECONOMÍA APLICADA’.}
\begin{itemize}
  \item Consulta:
  \begin{verbatim}
SQL> UPDATE ALU0101364544.DEPARTAMENTO
  2  SET D = 'ECONOMIA APLICADA'
  3  WHERE CD = 5;
  \end{verbatim}
  \item{Resultado:}
  \begin{verbatim}
1 row updated.
  \end{verbatim}
\end{itemize}

%Pregunta 8:
\chapter{Modificar en la tabla TITULACION la t-upla correspondiente a la titulación ‘GII’, poniendo como valor de facultad ‘ESIT1’}
\begin{itemize}
  \item Consulta:
  \begin{verbatim}
SQL> UPDATE ALU0101364544.TITULACION
  2  SET T = 'ESIT'
  3  WHERE T = 'GII';
  \end{verbatim}
  \item{Resultado:}
  \begin{verbatim}
UPDATE ALU0101364544.TITULACION
*
ERROR at line 1:
ORA-02292: integrity constraint (ALU0101364544.SYS_C00487529) violated - child 
record found 
  \end{verbatim}
\end{itemize}

%Pregunta 9:
\chapter{Eliminar en la tabla TITULACION la t-upla correspondiente a la titulación ‘GII’}
\begin{itemize}
  \item Consulta:
  \begin{verbatim}
SQL> DELETE FROM ALU0101364544.TITULACION
  2  WHERE T = 'GII';
  \end{verbatim}
  \item{Resultado:}
  \begin{verbatim}
DELETE FROM ALU0101364544.TITULACION
                          *
ERROR at line 1:
ORA-01031: insufficient privileges 
  \end{verbatim}
\end{itemize}

%Pregunta 10:
\chapter{Deshacer los cambios}
\begin{itemize}
  \item Consulta:
  \begin{verbatim}
SQL> ROLLBACK;
  \end{verbatim}
  \item{Resultado:}
  \begin{verbatim}
Rollback complete.
  \end{verbatim}
\end{itemize}

%Pregunta 11:
\chapter{Crear una vista, llamada ‘VISTA3’, sobre la tabla ‘ASIGNATURA’, con los atributos A y CAR}
\begin{itemize}
  \item Consulta:
  \begin{verbatim}
SQL> CREATE VIEW VISTA3 AS (
  2  SELECT A, CAR
  3  FROM ALU0101364544.ASIGNATURA);
  \end{verbatim}
  \item{Resultado:}
  \begin{verbatim}
View created.
  \end{verbatim}
\end{itemize}

%Pregunta 12:
\chapter{Eliminar la tabla ASIGNATURA}
\begin{itemize}
  \item Consulta:
  \begin{verbatim}
SQL> DROP TABLE ALU0101364544.ASIGNATURA;
  \end{verbatim}
  \item{Resultado:}
  \begin{verbatim}
DROP TABLE ALU0101364544.ASIGNATURA
                         *
ERROR at line 1:
ORA-01031: insufficient privileges 
  \end{verbatim}
\end{itemize}

%Pregunta 13:
\chapter{Listar las t-uplas de la tabla AREA}
\begin{itemize}
  \item Consulta:
  \begin{verbatim}
SQL> SELECT * FROM ALU0101364544.AREA;
  \end{verbatim}
  \item{Resultado:}
  \begin{verbatim}
       CAR AR                                                                   CD                                                                                                                      
---------- ------------------------------------------------------------ ----------                                                                                                                      
         1 ALGEBRA                                                               4                                                                                                                      
         2 ANALISIS MATEMATICO                                                   1                                                                                                                      
         3 ASTRONOMIA Y ASTROFISICA                                              2                                                                                                                      
         4 CIENCIAS DE LA COMPUTACION E INTELIGENCIA ARTIFICIAL                  3                                                                                                                      
         5 DIDACTICA DE LA MATEMATICA                                            1                                                                                                                      
         6 ESTADISTICA E INVESTIGACION OPERATIVA                                 3                                                                                                                      
         7 LENGUAJES Y SISTEMAS INFORMATICOS                                     3                                                                                                                      
         8 MATEMATICA APLICADA                                                   1                                                                                                                      

8 rows selected.
  \end{verbatim}
\end{itemize}

%Pregunta 14:
\chapter{Borrar en la tabla AREA la t-upla asociada al departamento con código 2}
\begin{itemize}
  \item Consulta:
  \begin{verbatim}
SQL> DELETE FROM ALU0101364544.AREA
  2  WHERE CD = 2;
  \end{verbatim}
  \item{Resultado:}
  \begin{verbatim}
1 row deleted.
  \end{verbatim}
\end{itemize}

%Pregunta 15:
\chapter{Borrar en la tabla AREA la t-upla asociada al departamento con código 3}
\begin{itemize}
  \item Consulta:
  \begin{verbatim}
SQL> DELETE FROM ALU0101364544.AREA
  2  WHERE CD = 3;
  \end{verbatim}
  \item{Resultado:}
  \begin{verbatim}
3 row deleted.
  \end{verbatim}
\end{itemize}

\newpage

%Pregunta 16:
\chapter{Insertar la t-upla (5, ‘ECONOMÍA APLICADA’) en la tabla DEPARTAMENTO}
\begin{itemize}
  \item Consulta:
  \begin{verbatim}
SQL> INSERT INTO ALU0101364544.DEPARTAMENTO
  2  VALUES(5, 'ECONOMIA APLICADA');
  \end{verbatim}
  \item{Resultado:}
  \begin{verbatim}
1 row created.
  \end{verbatim}
\end{itemize}

%Pregunta 17:
\chapter{Borrar las asignaturas adscritas al área con código 1}
\begin{itemize}
  \item Consulta:
  \begin{verbatim}
SQL> DELETE FROM ALU0101364544.ASIGNATURA
  2  WHERE CAR = 1;
  \end{verbatim}
  \item{Resultado:}
  \begin{verbatim}
DELETE FROM ALU0101364544.ASIGNATURA
                          *
ERROR at line 1:
ORA-01031: insufficient privileges 
  \end{verbatim}
\end{itemize}

\chapter{Eliminar en la tabla TITULACION la tupla correspondiente a la titulación ‘MII’}
\begin{itemize}
  \item Consulta:
  \begin{verbatim}
SQL> DELETE FROM ALU0101364544.TITULACION
  2  WHERE T = 'MII';
  \end{verbatim}
  \item{Resultado:}
  \begin{verbatim}
DELETE FROM ALU0101364544.TITULACION
                          *
ERROR at line 1:
ORA-01031: insufficient privileges 
  \end{verbatim}
\end{itemize}

\chapter{Hacer permanentes los cambios. Cerrar la sesión2}
\begin{itemize}
  \item Consulta:
  \begin{verbatim}
SQL> COMMIT WORK;
  \end{verbatim}
  \item{Resultado:}
  \begin{verbatim}
Commit complete.
  \end{verbatim}
\end{itemize}

\chapter{Cambiar a la sesión1}
Realizado.

\chapter{Listar las t-uplas de la tabla DEPARTAMENTO}
\begin{itemize}
  \item Consulta:
  \begin{verbatim}
SQL> SELECT * FROM DEPARTAMENTO;
  \end{verbatim}
  \item{Resultado:}
  \begin{verbatim}
        CD D                                                                    
---------- ------------------------------------------------------------         
         1 ANALISIS MATEMATICO                                                  
         2 ASTROFISICA                                                          
         3 ESTADISTICA, INVESTIGACION OPERATIVA Y COMPUTACION                   
         4 MATEMATICA FUNDAMENTAL                                               
         5 ECONOMIA APLICADA  
  \end{verbatim}
\end{itemize}

\chapter{Borrar en la tabla DEPARTAMENTO la t-upla asociada al departamento con código 2}
\begin{itemize}
  \item Consulta:
  \begin{verbatim}
SQL> DELETE FROM DEPARTAMENTO
  2  WHERE CD = 2;
  \end{verbatim}
  \item{Resultado:}
  \begin{verbatim}
1 row deleted.
  \end{verbatim}
\end{itemize}

\chapter{Listar las t-uplas de la tabla AREA}
\begin{itemize}
  \item Consulta:
  \begin{verbatim}
SQL> SELECT * FROM AREA;
  \end{verbatim}
  \item{Resultado:}
  \begin{verbatim}
       CAR AR                                                                   CD                                                                                                                      
---------- ------------------------------------------------------------ ----------                                                                                                                      
         1 ALGEBRA                                                               4                                                                                                                      
         2 ANALISIS MATEMATICO                                                   1                                                                                                                      
         5 DIDACTICA DE LA MATEMATICA                                            1                                                                                                                      
         8 MATEMATICA APLICADA                                                   1   
  \end{verbatim}
\end{itemize}

\chapter{Listar las t-uplas de la tabla PROFESOR}
\begin{itemize}
  \item Consulta:
  \begin{verbatim}
SQL> SELECT * FROM PROFESOR;
  \end{verbatim}
  \item{Resultado:}
  \begin{verbatim}
       DNI P                                                                   CAR CAT                                                                                                                  
---------- ------------------------------------------------------------ ---------- -----                                                                                                                
      4040 CARMELO                                                                 TU                                                                                                                   
      5050 CARINA                                                                8 CEU                                                                                                                  
      1111 JUAN                                                                    CU                                                                                                                   
      2222 CARLOS                                                                  TU                                                                                                                   
      3333 PEDRO                                                                   TEU                                                                                                                  
      4444 MARIA                                                                   TU                                                                                                                   
      5555 IVAN                                                                  1 CEU                                                                                                                  
      6666 CARMEN                                                                  CD                                                                                                                   
      7777 MARIO                                                                 2 TU                                                                                                                   
      8888 FRANCISCO                                                             5 TU                                                                                                                   
      9999 ANGELA                                                                8 TEU                                                                                                                  
      1010 DAVID                                                                   TU                                                                                                                   
      2020 SOLEDAD                                                                 CU                                                                                                                   
      3030 JOSE MANUEL                                                             TEU                                                                                                                  

14 rows selected.
  \end{verbatim}
\end{itemize}

\chapter{Listar las t-uplas de la tabla TITULACION}
\begin{itemize}
  \item Consulta:
  \begin{verbatim}
SQL> SELECT * FROM TITULACION;
  \end{verbatim}
  \item{Resultado:}
  \begin{verbatim}
T    FAC                                                                                                                                                                                                
---- ------------------------------------------------------------                                                                                                                                       
GF   MATEMATICAS Y FISICA                                                                                                                                                                               
GM   MATEMATICAS Y FISICA                                                                                                                                                                               
GII  ESIT                                                                                                                                                                                               
MII  ESIT         
  \end{verbatim}
\end{itemize}

% Pregunta 26:
\chapter{Quitar todos los privilegios concedidos al usuario alumno sobre la tabla DEPARTAMENTO.}
\begin{itemize}
  \item Consulta:
  \begin{verbatim}
SQL> REVOKE ALL
  2  ON DEPARTAMENTO
  3  FROM ALUMNO;
  \end{verbatim}
  \item{Resultado:}
  \begin{verbatim}
Revoke succeeded.
  \end{verbatim}
\end{itemize}

% Pregunta 27:
\chapter{Quitar todos los privilegios concedidos al usuario alumno}
\begin{itemize}
  \item Consulta:
  \begin{verbatim}
SQL> REVOKE ALL
  2  FROM ALUMNO;
  \end{verbatim}
  \item{Resultado:}
  \begin{verbatim}
Revoke succeeded.
  \end{verbatim}
\end{itemize}

\chapter{Hacer permanentes los cambios}
\begin{itemize}
  \item Consulta:
  \begin{verbatim}
SQL> COMMIT WORK;
  \end{verbatim}
  \item{Resultado:}
  \begin{verbatim}
Commit complete.
  \end{verbatim}
\end{itemize}

\end{document}