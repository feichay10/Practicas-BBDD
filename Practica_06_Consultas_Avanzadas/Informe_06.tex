\documentclass[11pt]{report}

% Paquetes y configuraciones adicionales
\usepackage{graphicx}
\usepackage[export]{adjustbox}
\usepackage{caption}
\usepackage{float}
\usepackage{titlesec}
\usepackage{geometry}
\usepackage[hidelinks]{hyperref}
\usepackage{titling}
\usepackage{titlesec}
\usepackage{parskip}
\usepackage{wasysym}
\usepackage{tikzsymbols}
\usepackage{fancyvrb}
\usepackage{xurl}
\usepackage{hyperref}
\usepackage{subcaption}

\usepackage{listings}
\usepackage{xcolor}

\usepackage[spanish]{babel}

\newcommand{\subtitle}[1]{
  \posttitle{
    \par\end{center}
    \begin{center}\large#1\end{center}
    \vskip0.5em}
}

% Configura los márgenes
\geometry{
  left=2cm,   % Ajusta este valor al margen izquierdo deseado
  right=2cm,  % Ajusta este valor al margen derecho deseado
  top=3cm,
  bottom=3cm,
}

% Configuración de los títulos de las secciones
\titlespacing{\section}{0pt}{\parskip}{\parskip}
\titlespacing{\subsection}{0pt}{\parskip}{\parskip}
\titlespacing{\subsubsection}{0pt}{\parskip}{\parskip}

% Redefinir el formato de los capítulos y añadir un punto después del número
\makeatletter
\renewcommand{\@makechapterhead}[1]{%
  \vspace*{0\p@} % Ajusta este valor para el espaciado deseado antes del título del capítulo
  {\parindent \z@ \raggedright \normalfont
    \ifnum \c@secnumdepth >\m@ne
        \huge\bfseries \thechapter.\ % Añade un punto después del número
    \fi
    \interlinepenalty\@M
    #1\par\nobreak
    \vspace{10pt} % Ajusta este valor para el espacio deseado después del título del capítulo
  }}
\makeatother

% Configura para que cada \chapter no comience en una pagina nueva
\makeatletter
\renewcommand\chapter{\@startsection{chapter}{0}{\z@}%
    {-3.5ex \@plus -1ex \@minus -.2ex}%
    {2.3ex \@plus.2ex}%
    {\normalfont\Large\bfseries}}
\makeatother

% Configurar los colores para el código
\definecolor{codegreen}{rgb}{0,0.6,0}
\definecolor{codegray}{rgb}{0.5,0.5,0.5}
\definecolor{codepurple}{rgb}{0.58,0,0.82}
\definecolor{backcolour}{rgb}{0.95,0.95,0.92}

% Configurar el estilo para el código
\lstdefinestyle{mystyle}{
  backgroundcolor=\color{backcolour},   
  commentstyle=\color{codegreen},
  keywordstyle=\color{magenta},
  numberstyle=\tiny\color{codegray},
  stringstyle=\color{codepurple},
  basicstyle=\ttfamily\footnotesize,
  breakatwhitespace=false,         
  breaklines=true,                 
  captionpos=b,                    
  keepspaces=true,                 
  numbers=left,                    
  numbersep=5pt,                  
  showspaces=false,                
  showstringspaces=false,
  showtabs=false,                  
  tabsize=2
}

\begin{document}

% Portada del informe
\title{Practica 06. Consultas avanzadas}
\subtitle{Bases de Datos}
\author{Cheuk Kelly Ng Pante (alu0101364544@ull.edu.es)}
\date{\today}

\maketitle

\pagestyle{empty} % Desactiva la numeración de página para el índice

% Índice
\tableofcontents

% Nueva página
\cleardoublepage

\pagestyle{plain} % Vuelve a activar la numeración de página
\setcounter{page}{1} % Reinicia el contador de página a 1

% Secciones del informe
% Capitulo 1
\chapter{Obtener la fecha del sistema.}
\chapter{Obtener la hora del sistema.}
\chapter{Dar la fecha del sistema con el formato día de la semana, día del mes, mes y año.}
\chapter{Dar la hora del sistema en formato de reloj de 24 horas.}
\chapter{Obtener el número de días que lleva impartiendo la asignatura con código 11 el profesor con DNI 8888.}
\chapter{Listar los nombres de profesores que han impartido una asignatura más de 365 días.}
\chapter{Hallar el número de profesores del departamento ‘ASTROFÍSICA’.}
\chapter{Hallar para cada departamento el número de profesores que tiene. Ordena la salida alfabéticamente.}
\chapter{Hallar en cuantas titulaciones imparte el departamento de ‘ESTADÍSTICA, INVESTIGACIÓN OPERATIVA Y COMPUTACIÓN’.}
\chapter{Hallar el número de profesores adscritos a áreas cuyo nombre (el de las áreas) empiece por ‘A’.}
\chapter{Hallar para cada titulación el número de asignaturas que tiene. Ordena la salida alfabéticamente.}
\chapter{Listar el nombre de la asignatura con más créditos teóricos.}
\chapter{Listar el nombre de la asignatura con menos créditos teóricos.}
\chapter{Listar para cada asignatura el número total de créditos que tiene.}
\chapter{Listar el nombre de la asignatura con más créditos.}
\chapter{Listar el nombre de la asignatura con menos créditos.}
\chapter{Listar el nombre del área a la que está adscrita la asignatura con más créditos.}
\chapter{Hallar el número de asignaturas impartidas por el profesor con DNI 1111.}
\chapter{Hallar el número de créditos impartidos por el profesor con DNI 1111.}
\chapter{Hallar el nombre del profesor que más créditos imparte actualmente.}
\chapter{Hallar el número medio de asignaturas adscritas a cada área.}
\chapter{Hallar el número medio de profesores de cada departamento.}
\chapter{Hallar los nombres de las áreas que tengan más de 3 asignaturas.}
\chapter{Hallar los nombres de las áreas que tengan más de 6 asignaturas.}
\chapter{Hallar el nombre del departamento con menos profesores.}


\end{document}