\documentclass[11pt]{report}

% Paquetes y configuraciones adicionales
\usepackage{graphicx}
\usepackage[export]{adjustbox}
\usepackage{caption}
\usepackage{float}
\usepackage{titlesec}
\usepackage{geometry}
\usepackage[hidelinks]{hyperref}
\usepackage{titling}
\usepackage{titlesec}
\usepackage{parskip}
\usepackage{wasysym}
\usepackage{tikzsymbols}
\usepackage{fancyvrb}
\usepackage{xurl}
\usepackage{hyperref}
\usepackage{subcaption}

\usepackage{listings}
\usepackage{xcolor}

\usepackage[spanish]{babel}

\newcommand{\subtitle}[1]{
  \posttitle{
    \par\end{center}
    \begin{center}\large#1\end{center}
    \vskip0.5em}
}

% Configura los márgenes
\geometry{
  left=2cm,   % Ajusta este valor al margen izquierdo deseado
  right=2cm,  % Ajusta este valor al margen derecho deseado
  top=3cm,
  bottom=3cm,
}

% Configuración de los títulos de las secciones
\titlespacing{\section}{0pt}{\parskip}{\parskip}
\titlespacing{\subsection}{0pt}{\parskip}{\parskip}
\titlespacing{\subsubsection}{0pt}{\parskip}{\parskip}

% Redefinir el formato de los capítulos y añadir un punto después del número
\makeatletter
\renewcommand{\@makechapterhead}[1]{%
  \vspace*{0\p@} % Ajusta este valor para el espaciado deseado antes del título del capítulo
  {\parindent \z@ \raggedright \normalfont
    \ifnum \c@secnumdepth >\m@ne
        \huge\bfseries \thechapter.\ % Añade un punto después del número
    \fi
    \interlinepenalty\@M
    #1\par\nobreak
    \vspace{10pt} % Ajusta este valor para el espacio deseado después del título del capítulo
  }}
\makeatother

% Configura para que cada \chapter no comience en una pagina nueva
\makeatletter
\renewcommand\chapter{\@startsection{chapter}{0}{\z@}%
    {-3.5ex \@plus -1ex \@minus -.2ex}%
    {2.3ex \@plus.2ex}%
    {\normalfont\Large\bfseries}}
\makeatother

% Configurar los colores para el código
\definecolor{codegreen}{rgb}{0,0.6,0}
\definecolor{codegray}{rgb}{0.5,0.5,0.5}
\definecolor{codepurple}{rgb}{0.58,0,0.82}
\definecolor{backcolour}{rgb}{0.95,0.95,0.92}

% Configurar el estilo para el código
\lstdefinestyle{mystyle}{
  backgroundcolor=\color{backcolour},   
  commentstyle=\color{codegreen},
  keywordstyle=\color{magenta},
  numberstyle=\tiny\color{codegray},
  stringstyle=\color{codepurple},
  basicstyle=\ttfamily\footnotesize,
  breakatwhitespace=false,         
  breaklines=true,                 
  captionpos=b,                    
  keepspaces=true,                 
  numbers=left,                    
  numbersep=5pt,                  
  showspaces=false,                
  showstringspaces=false,
  showtabs=false,                  
  tabsize=2
}

\begin{document}

% Portada del informe
\title{Practica 06. Consultas avanzadas}
\subtitle{Bases de Datos}
\author{Cheuk Kelly Ng Pante (alu0101364544@ull.edu.es)}
\date{\today}

\maketitle

\pagestyle{empty} % Desactiva la numeración de página para el índice

% Índice
\tableofcontents

% Nueva página
\cleardoublepage

\pagestyle{plain} % Vuelve a activar la numeración de página
\setcounter{page}{1} % Reinicia el contador de página a 1

% Secciones del informe
% Capitulo 1
\chapter{Obtener la fecha del sistema.}
\begin{itemize}
  \item Consulta:
  \begin{verbatim}
SQL> SELECT SYSDATE
2  FROM DUAL;  
  \end{verbatim}
  \item{Resultado:}
  \begin{verbatim}
SYSDATE                                                                         
---------                                                                       
14-DEC-24                                                                                                                                         
  \end{verbatim}
\end{itemize}

% Capitulo 2
\chapter{Obtener la hora del sistema.}
\begin{itemize}
  \item Consulta:
  \begin{verbatim}
SQL> SELECT TO_CHAR(SYSDATE, 'HH-MI-SS') AS HORA
  2  FROM DUAL;
  \end{verbatim}
  \item{Resultado:}
  \begin{verbatim}
HORA                                                                            
--------                                                                        
04-21-45                                                                                
  \end{verbatim}
\end{itemize}

% Capitulo 3
\chapter{Dar la fecha del sistema con el formato día de la semana, día del mes, mes y año.}
\begin{itemize}
  \item Consulta:
  \begin{verbatim}
SQL> SELECT TO_CHAR(SYSDATE, 'DAY DD-MM-YY') AS FECHA
  2  FROM DUAL;
  \end{verbatim}
  \item{Resultado:}
  \begin{verbatim}
FECHA                                                                           
---------------------------------------------                                   
SATURDAY  14-12-24                                                                                                                                            
  \end{verbatim}
\end{itemize}

\newpage

% Capitulo 4
\chapter{Dar la hora del sistema en formato de reloj de 24 horas.}
\begin{itemize}
  \item Consulta:
  \begin{verbatim}
SQL> SELECT TO_CHAR(SYSDATE, 'HH24:MM') AS HORA_24
  2  FROM DUAL;
  \end{verbatim}
  \item{Resultado:}
  \begin{verbatim}
HORA_                                                                           
-----                                                                           
16:12                                                                             
  \end{verbatim}
\end{itemize}

% Capitulo 5
\chapter{Obtener el número de días que lleva impartiendo la asignatura con código 11 el profesor con DNI 8888.}
\begin{itemize}
  \item Consulta:
  \begin{verbatim}
SQL> SELECT SYSDATE - FI
  2  FROM PLAN_DOCENTE
  3  WHERE CAS = 11 AND DNI = 8888;
  \end{verbatim}
  \item{Resultado:}
  \begin{verbatim}
SYSDATE-FI                                                                      
----------                                                                      
5583.68457                                                                      
  \end{verbatim}
\end{itemize}

% Capitulo 6
\chapter{Listar los nombres de profesores que han impartido una asignatura más de 365 días.}
\begin{itemize}
  \item Consulta:
  \begin{verbatim}
SQL> SELECT P
  2  FROM PROFESOR
  3  WHERE DNI IN
  4  (SELECT DNI FROM PLAN_DOCENTE
  5  WHERE FF - FI > 365);
  \end{verbatim}

  \newpage

  \item{Resultado:}
  \begin{verbatim}
P                                                                               
------------------------------------------------------------                    
JUAN                                                                            
MARIA                                                                           
CARMEN                                                                          
DAVID                                                                          
  \end{verbatim}
\end{itemize}

% Capitulo 7
\chapter{Hallar el número de profesores del departamento ‘ASTROFÍSICA’.}
\begin{itemize}
  \item Consulta:
  \begin{verbatim}
SQL> SELECT COUNT(P)
  2  FROM DEPARTAMENTO NATURAL JOIN AREA NATURAL JOIN PROFESOR
  3  WHERE D = 'ASTROFISICA';
  \end{verbatim}
  \item{Resultado:}
  \begin{verbatim}
  COUNT(P)                                                                      
----------                                                                      
         1                                                                            
  \end{verbatim}
\end{itemize}

% Capitulo 8
\chapter{Hallar para cada departamento el número de profesores que tiene. Ordena la salida alfabéticamente.}
\begin{itemize}
  \item Consulta:
  \begin{verbatim}
SQL> SELECT D, COUNT(P)
  2  FROM DEPARTAMENTO NATURAL JOIN AREA NATURAL JOIN PROFESOR
  3  GROUP BY D
  4  ORDER BY D;
  \end{verbatim}
  \item{Resultado:}
  \begin{verbatim}
D                                                              COUNT(P)         
------------------------------------------------------------ ----------         
ANALISIS MATEMATICO                                                   3         
ASTROFISICA                                                           1         
ESTADISTICA, INVESTIGACION OPERATIVA Y COMPUTACION                    7         
MATEMATICA FUNDAMENTAL                                                1                                                                     
  \end{verbatim}
\end{itemize}

% Capitulo 9
\chapter{Hallar en cuantas titulaciones imparte el departamento de ‘ESTADÍSTICA, INVESTIGACIÓN OPERATIVA Y COMPUTACIÓN’.}
\begin{itemize}
  \item Consulta:
  \begin{verbatim}
SQL> SELECT COUNT(T)
  2  FROM DEPARTAMENTO NATURAL JOIN AREA NATURAL JOIN ASIGNATURA
  3  WHERE D = 'ESTADISTICA, INVESTIGACION OPERATIVA Y COMPUTACION';
  \end{verbatim}
  \item{Resultado:}
  \begin{verbatim}
  COUNT(T)                                                                      
----------                                                                      
         7                                                                        
  \end{verbatim}
\end{itemize}

% Capitulo 10
\chapter{Hallar el número de profesores adscritos a áreas cuyo nombre (el de las áreas) empiece por ‘A’.}
\begin{itemize}
  \item Consulta:
  \begin{verbatim}
SQL> SELECT COUNT(P)
  2  FROM AREA NATURAL JOIN PROFESOR
  3  WHERE AR LIKE 'A%';
  \end{verbatim}
  \item{Resultado:}
  \begin{verbatim}
   COUNT(P)                                                                      
----------                                                                      
         3                                                                               
  \end{verbatim}
\end{itemize}

% Capitulo 11
\chapter{Hallar para cada titulación el número de asignaturas que tiene. Ordena la salida alfabéticamente.}
\begin{itemize}
  \item Consulta:
  \begin{verbatim}
SQL> SELECT T, COUNT(CAS)
  2  FROM ASIGNATURA
  3  GROUP BY T
  4  ORDER BY T;
  \end{verbatim}

  \newpage

  \item{Resultado:}
  \begin{verbatim}
T    COUNT(CAS)                                                                 
---- ----------                                                                 
GF            1                                                                 
GII           7                                                                 
GM            2                                                                 
MII           2                                                                               
  \end{verbatim}
\end{itemize}

% Capitulo 12
\chapter{Listar el nombre de la asignatura con más créditos teóricos.}
\begin{itemize}
  \item Consulta:
  \begin{verbatim}
SQL> SELECT A
  2  FROM ASIGNATURA
  3  WHERE CT IN
  4  (SELECT MAX(CT) FROM ASIGNATURA);
  \end{verbatim}
  \item{Resultado:}
  \begin{verbatim}
A                                                                               
--------------------------------------------------                              
DIDACTICA DE LA MATEMATICA                                                                                
  \end{verbatim}
\end{itemize}

% Capitulo 13
\chapter{Listar el nombre de la asignatura con menos créditos teóricos.}
\begin{itemize}
  \item Consulta:
  \begin{verbatim}
SQL> SELECT A
  2  FROM ASIGNATURA
  3  WHERE CT IN
  4  (SELECT MIN(CT)
  5  FROM ASIGNATURA);
  \end{verbatim}
  \item{Resultado:}
  \begin{verbatim}
A                                                                               
--------------------------------------------------                              
INTELIGENCIA ARTIFICIAL                                                         
ALMACENES DE DATOS                                                              
MINERIA DE DATOS                                                                                 
  \end{verbatim}
\end{itemize}

% Capitulo 14
\chapter{Listar para cada asignatura el número total de créditos que tiene.}
\begin{itemize}
  \item Consulta:
  \begin{verbatim}
SQL> SELECT A, CT+CP+CL AS "CREDITOS TOTALES"
  2  FROM ASIGNATURA;
  \end{verbatim}
  \item{Resultado:}
  \begin{verbatim}
A                                                  CREDITOS TOTALES             
-------------------------------------------------- ----------------             
BASES DE DATOS                                                    6             
INTELIGENCIA ARTIFICIAL                                           6             
ALMACENES DE DATOS                                                3             
MINERIA DE DATOS                                                  3             
INFORMATICA BASICA                                                6             
ALGEBRA                                                           6             
CALCULO                                                           6             
OPTIMIZACION                                                      6             
GESTION DE RIESGOS                                                6             
ASTRONOMIA                                                        6             
DIDACTICA DE LA MATEMATICA                                        6             

A                                                  CREDITOS TOTALES             
-------------------------------------------------- ----------------             
ANALISIS COMPLEJO                                               7.5             

12 rows selected.                                                                            
  \end{verbatim}
\end{itemize}

% Capitulo 15
\chapter{Listar el nombre de la asignatura con más créditos.}
\begin{itemize}
  \item Consulta:
  \begin{verbatim}
SQL> SELECT A
  2  FROM ASIGNATURA
  3  WHERE CT+CP+CL = (SELECT MAX(CT+CP+CL)
  4  FROM ASIGNATURA);
  \end{verbatim}
  \item{Resultado:}
  \begin{verbatim}
A                                                                               
--------------------------------------------------                              
ANALISIS COMPLEJO                                                                        
  \end{verbatim}
\end{itemize}

% Capitulo 16
\chapter{Listar el nombre de la asignatura con menos créditos.}
\begin{itemize}
  \item Consulta:
  \begin{verbatim}
SQL> SELECT A
  2  FROM ASIGNATURA
  3  WHERE CT+CP+CL = (SELECT MIN(CT+CP+CL)
  4  FROM ASIGNATURA);
  \end{verbatim}
  \item{Resultado:}
  \begin{verbatim}
A                                                                               
--------------------------------------------------                              
ALMACENES DE DATOS                                                              
MINERIA DE DATOS                                                                    
  \end{verbatim}
\end{itemize}

% Capitulo 17
\chapter{Listar el nombre del área a la que está adscrita la asignatura con más créditos.}
\begin{itemize}
  \item Consulta:
  \begin{verbatim}
SQL> SELECT AR
  2  FROM AREA NATURAL JOIN ASIGNATURA
  3  WHERE CP+CT+CL = (SELECT MAX(CP+CT+CL)
  4  FROM ASIGNATURA);
  \end{verbatim}
  \item{Resultado:}
  \begin{verbatim}
AR                                                                              
------------------------------------------------------------                    
ANALISIS MATEMATICO                                                                               
  \end{verbatim}
\end{itemize}

% Capitulo 18
\chapter{Hallar el número de asignaturas impartidas por el profesor con DNI 1111.}
\begin{itemize}
  \item Consulta:
  \begin{verbatim}
SQL> SELECT COUNT(A)
  2  FROM ASIGNATURA NATURAL JOIN PROFESOR
  3  WHERE DNI = 1111;
  \end{verbatim}

  \newpage

  \item{Resultado:}
  \begin{verbatim}
  COUNT(A)                                                                      
----------                                                                      
         1                                                                     
  \end{verbatim}
\end{itemize}

% Capitulo 19
\chapter{Hallar el número de créditos impartidos por el profesor con DNI 1111.}
\begin{itemize}
  \item Consulta:
  \begin{verbatim}
SQL> SELECT SUM(CTA+CPA+CLA) CREDITOS_1111
  2  FROM PLAN_DOCENTE
  3  WHERE DNI = 1111;
  \end{verbatim}
  \item{Resultado:}
  \begin{verbatim}
CREDITOS_1111                                                                   
-------------                                                                   
            9                                                                         
  \end{verbatim}
\end{itemize}

% Capitulo 20
\chapter{Hallar el nombre del profesor que más créditos imparte actualmente.}
\begin{itemize}
  \item Consulta:
  \begin{verbatim}
SQL> SELECT P
  2  FROM PROFESOR NATURAL JOIN PLAN_DOCENTE
  3  WHERE CPA+CTA+CLA = (SELECT MAX(CPA+CTA+CLA)
  4  FROM PLAN_DOCENTE
  5  WHERE FF IS NULL);
  \end{verbatim}
  \item{Resultado:}
  \begin{verbatim}
P                                                                               
------------------------------------------------------------                    
MARIO                                                                                 
  \end{verbatim}
\end{itemize}

\newpage

% Capitulo 21
\chapter{Hallar el número medio de asignaturas adscritas a cada área.}
\begin{itemize}
  \item Consulta:
  \begin{verbatim}
SQL> SELECT AVG(COUNT(P))
  2  FROM DEPARTAMENTO NATURAL JOIN AREA NATURAL JOIN PROFESOR
  3  GROUP BY D;
  \end{verbatim}
  \item{Resultado:}
  \begin{verbatim}
     MEDIA                                                                      
----------                                                                      
       1.5                                                                            
  \end{verbatim}
\end{itemize}

% Capitulo 22
\chapter{Hallar el número medio de profesores de cada departamento.}
\begin{itemize}
  \item Consulta:
  \begin{verbatim}
SQL> SELECT AVG(COUNT(P)) AS "NUM MEDIO PROFE DEPARTAMENTO"
  2  FROM DEPARTAMENTO NATURAL JOIN AREA NATURAL JOIN PROFESOR
  3  GROUP BY D;
  \end{verbatim}
  \item{Resultado:}
  \begin{verbatim}
NUM MEDIO PROFE DEPARTAMENTO                                                    
----------------------------                                                    
                           3                                                                               
  \end{verbatim}
\end{itemize}

% Capitulo 23
\chapter{Hallar los nombres de las áreas que tengan más de 3 asignaturas.}
\begin{itemize}
  \item Consulta:
  \begin{verbatim}
SQL> SELECT AR
  2  FROM AREA NATURAL JOIN ASIGNATURA
  3  GROUP BY AR
  4  HAVING COUNT(A) > 3;
  \end{verbatim}
  \item{Resultado:}
  \begin{verbatim}
AR                                                                              
------------------------------------------------------------                    
LENGUAJES Y SISTEMAS INFORMATICOS                                                                        
  \end{verbatim}
\end{itemize}

\chapter{Hallar los nombres de las áreas que tengan más de 6 asignaturas.}
\begin{itemize}
  \item Consulta:
  \begin{verbatim}
SQL> SELECT AR
  2  FROM AREA NATURAL JOIN ASIGNATURA
  3  GROUP BY AR
  4  HAVING COUNT(A) > 6;
  \end{verbatim}
  \item{Resultado:}
  \begin{verbatim}
AR                                                                              
------------------------------------------------------------                    
LENGUAJES Y SISTEMAS INFORMATICOS                                                                               
  \end{verbatim}
\end{itemize}

\chapter{Hallar el nombre del departamento con menos profesores.}
\begin{itemize}
  \item Consulta:
  \begin{verbatim}
SQL> SELECT D
  2  FROM DEPARTAMENTO NATURAL JOIN AREA NATURAL JOIN PROFESOR
  3  HAVING COUNT(P) = (SELECT MIN(COUNT(P))
  4  FROM DEPARTAMENTO NATURAL JOIN AREA NATURAL JOIN PROFESOR
  5  GROUP BY D)
  6  GROUP BY D;
  \end{verbatim}
  \item{Resultado:}
  \begin{verbatim}
D                                                                               
------------------------------------------------------------                    
ASTROFISICA                                                                     
MATEMATICA FUNDAMENTAL                                                                         
  \end{verbatim}
\end{itemize}

\end{document}