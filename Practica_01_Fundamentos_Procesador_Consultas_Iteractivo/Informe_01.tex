\documentclass[11pt]{report}

% Paquetes y configuraciones adicionales
\usepackage{graphicx}
\usepackage[export]{adjustbox}
\usepackage{caption}
\usepackage{float}
\usepackage{titlesec}
\usepackage{geometry}
\usepackage[hidelinks]{hyperref}
\usepackage{titling}
\usepackage{titlesec}
\usepackage{parskip}
\usepackage{wasysym}
\usepackage{tikzsymbols}
\usepackage{fancyvrb}
\usepackage{xurl}
\usepackage{hyperref}
\usepackage{subcaption}

\usepackage{listings}
\usepackage{xcolor}

\usepackage[spanish]{babel}

\newcommand{\subtitle}[1]{
  \posttitle{
    \par\end{center}
    \begin{center}\large#1\end{center}
    \vskip0.5em}
}

% Configura los márgenes
\geometry{
  left=2cm,   % Ajusta este valor al margen izquierdo deseado
  right=2cm,  % Ajusta este valor al margen derecho deseado
  top=3cm,
  bottom=3cm,
}

% Configuración de los títulos de las secciones
\titlespacing{\section}{0pt}{\parskip}{\parskip}
\titlespacing{\subsection}{0pt}{\parskip}{\parskip}
\titlespacing{\subsubsection}{0pt}{\parskip}{\parskip}

% Redefinir el formato de los capítulos y añadir un punto después del número
\makeatletter
\renewcommand{\@makechapterhead}[1]{%
  \vspace*{0\p@} % Ajusta este valor para el espaciado deseado antes del título del capítulo
  {\parindent \z@ \raggedright \normalfont
    \ifnum \c@secnumdepth >\m@ne
        \huge\bfseries \thechapter.\ % Añade un punto después del número
    \fi
    \interlinepenalty\@M
    #1\par\nobreak
    \vspace{10pt} % Ajusta este valor para el espacio deseado después del título del capítulo
  }}
\makeatother

% Configura para que cada \chapter no comience en una pagina nueva
\makeatletter
\renewcommand\chapter{\@startsection{chapter}{0}{\z@}%
    {-3.5ex \@plus -1ex \@minus -.2ex}%
    {2.3ex \@plus.2ex}%
    {\normalfont\Large\bfseries}}
\makeatother

% Configurar los colores para el código
\definecolor{codegreen}{rgb}{0,0.6,0}
\definecolor{codegray}{rgb}{0.5,0.5,0.5}
\definecolor{codepurple}{rgb}{0.58,0,0.82}
\definecolor{backcolour}{rgb}{0.95,0.95,0.92}

% Configurar el estilo para el código
\lstdefinestyle{mystyle}{
  backgroundcolor=\color{backcolour},   
  commentstyle=\color{codegreen},
  keywordstyle=\color{magenta},
  numberstyle=\tiny\color{codegray},
  stringstyle=\color{codepurple},
  basicstyle=\ttfamily\footnotesize,
  breakatwhitespace=false,         
  breaklines=true,                 
  captionpos=b,                    
  keepspaces=true,                 
  numbers=left,                    
  numbersep=5pt,                  
  showspaces=false,                
  showstringspaces=false,
  showtabs=false,                  
  tabsize=2
}

\begin{document}

% Portada del informe
\title{Practica 01. Fundamentos procesador de consultas interactivo}
\subtitle{Bases de Datos}
\author{Cheuk Kelly Ng Pante (alu0101364544@ull.edu.es)}
\date{\today}

\maketitle

\pagestyle{empty} % Desactiva la numeración de página para el índice

% Índice
\tableofcontents

% Nueva página
\cleardoublepage

\pagestyle{plain} % Vuelve a activar la numeración de página
\setcounter{page}{1} % Reinicia el contador de página a 1

% Secciones del informe
% Capitulo 1
\chapter{Gestión sesión}
\begin{itemize}
  \item \textbf{CONNECT:} Conecta al usuario con una base de datos.
  \item \textbf{PASSWORD:} Cambia la contraseña del usuario.
  \item \textbf{DISCONNECT:} Desconecta al usuario de la base de datos.
  \item \textbf{EXIT:} Cierra la aplicación.
\end{itemize}

% Capitulo 2
\chapter{Gestión buffer de instrucciones}
\begin{itemize}
  \item \textbf{LIST:} Enumera una o más líneas del comando SQL o bloque ejecutado más recientemente que se almacena en el buffer SQL.
  \item \textbf{INPUT:} Añade una o más líneas nuevas de texto después de la línea actual en el buffer.
  \item \textbf{APPEND:} Agrega el texto especificado al final de la línea actual en el buffer SQL.
  \item \textbf{CHANGE:} Cambia la primera aparición de antigua en la línea actual del buffer SQL.
  \item \textbf{DEL:} elimina una o más líneas del buffer.
  \item \textbf{SAVE:} guarda el contenido del buffer SQL en un archivo.
  \item \textbf{GET:} carga una instrucción SQL o un bloque PL / SQL de un archivo al buffer SQL.
  \item \textbf{RUN:}  enumera y ejecuta el comando SQL ejecutado más recientemente o el bloque PL / SQL que se almacena en el buffer SQL.
\end{itemize}

% Capitulo 3
\chapter{Gestión de scripts}
\begin{itemize}
  \item \textbf{EDIT:} Para editar el contenido del buffer SQL.
  \item \textbf{REMARK:} Hace un comentario en un script.
  \item \textbf{START:} Ejecuta las sentencias SQL Plus en el script especificado.
  \item \textbf{COPY:} Copia datos de una consulta a una tabla en la misma base de datos o en otra.
\end{itemize}

% Capitulo 4
\chapter{Gestión del entorno SQL Plus}
\begin{itemize}
  \item \textbf{DESCRIBE:} Enumera las definiciones de columna para una tabla, vista o sinónimo, o las especificaciones para un función o procedimiento.
  \item \textbf{SPOOL:} Almacena los resultados de la consulta en un archivo.
  \item \textbf{SHOW:} Muestra el valor de una variable del sistema SQLPlus o el entorno actual.
  \item \textbf{SET:} Establece una variable de sistema para alterar la configuración de SQLPlus para su sesión.
  \item \textbf{HELP:} Accede al sistema de ayuda de la línea de comandos.
  \item \textbf{HOST:} Ejecuta un comando del sistema operativo sin salir de SQLPlus.
\end{itemize}

% Capitulo 5
\chapter{Gestión de variables}
\begin{itemize}
  \item \textbf{DEFINE:} Especifica una variable de sustitución y le asigna un valor CHAR, o enumera el valor y tipo de variable de una sola variable o de todas las variables.
  \item \textbf{UNDEFINE:} Elimina una o más variables de sustitución que se definió
  \item \textbf{VARIABLE:} Declara una variable de vinculación a la que se puede hacer referencia en PL / SQL, o enumera la pantalla actual características para una sola variable o todas las variables.
  \item \textbf{PRINT:} Muestra los valores actuales de las variables de vinculación o enumera todas las variables de vinculación
\end{itemize}

% Capitulo 6
\chapter{Gestión de informes}
\begin{itemize}
  \item \textbf{PAUSE:} Muestra el texto especificado y luego espera a que el usuario presione RETORNO.
  \item \textbf{TITLE:} Coloca y formatea un título específico en la parte superior de cada página del informe.
  \item \textbf{BITTLE:} Coloca y formatea un título en la parte inferior de cada página del informe.
  \item \textbf{COLUMN:} Especifica los atributos de visualización para una columna determinada.
\end{itemize}

\end{document}